% !TeX root = ../thuthesis-example.tex

\chapter{按数位顺序进行的动态规划}

本部分将介绍按数位顺序进行的动态规划(后文简称为数位DP)

\section{适用问题类型}

给定区间 \([L,R]\),求区间内满足一定条件的数的个数。

问题有如下特征:

\begin{itemize}
\item
  给定的条件一般与数的组成有关,比如数位之间的关系。
\item
  此类问题的数据规模一般较大。\(1\leq L<R\leq 10^{100000}\)
  在某些情况下也是可以接受的,因为是按位DP。
\item
  最终目的为计数,统计满足某一性质的数的个数。
\end{itemize}

为了方便,数位DP一般使用记忆化搜索实现。

\section{算法简述}

通常情况下,题目可转化为求区间 \([L, R]\) 内满足性质 \(p\)
的数的个数。利用前缀和的思想,先将求 \(\operatorname{ans}_{[L,R]}\)
的问题转化为
\(\operatorname{ans}_{[1,R]}-\operatorname{ans}_{[1,l-1]}\).

数位DP的核心思想,是基于:在从 \(1\) 开始逐个向更大的数遍历的过程中,有很多性质重复的部分。

比如:求 \(1\sim 10000\) 中,存在某一数位为 \(7\) 的数的个数。在
\(3000\sim3999\),\(4000\sim4999\),\(5000\sim 5999\)
中,它们的后三位都是 \(\_000\sim \_ 999\),是完全相同的,因此
\(\operatorname{ans}_{[3000,3999]}=\operatorname{ans}_{[4000,4999]}=\operatorname{ans}_{[5000,5999]}\),对这些部分加以记录,相同的部分只计算一次,可有效减少计算量。这与动态规划利用数组记录已经统计过的部分,相同的部分不再重复计算的性质一致。

统计答案的过程再加上记忆过程,这就是记忆化搜索,与动态规划本质无差别,因而算法名为数位
DP。

\section{应用实例}

\subsection{题目来源}

题目名称:windy 数。

题目选自:NOI2009四川省省队选拔活动。

\subsection{题目描述}

定义 \(\rm{windy}\) 数:不含前导零且相邻两个数字之差至少为 \(2\)
的正整数。

给定 \(a,b\),求 \([a,b]\) 中有多少个 \(\rm{windy}\) 数。

数据范围:\(1\leq a\leq b\leq 2\times 10^9\).

\subsection{解题思路}

设 \(\operatorname{ans}_{[a,b]}\) 代表题目所求,把
\(\operatorname{ans}_{[a,b]}\) 化为
\(\operatorname{ans}_{[a,b]}=\operatorname{ans}_{[1,b]}-\operatorname{ans}_{[1,a-1]}\).

下面求 \(\operatorname{ans}_{[1,r]}\),称 \(r\) 为上界。

定义四元组为动态规划状态:\(\rm{(pos,lst,lim,zero)}\),代表的含义分别为:

\begin{itemize}
\item
  \(\rm{int\;pos}\):当前正在尝试确定第几位,个位是第 \(1\) 位,十位是第
  \(2\) 位,以此类推。
\item
  \(\rm{int \;lst}\):上一位选取的数,其用于约束相邻两个数字之差不超过 \(2\)。
\item
  \(\rm{bool\;lim}\):当前选出的前缀是否与上界相同,其用于约束当前数位的选取使其不超过上界。比如上界为
  \(273967\),当前枚举出的前缀为 \(2739\underline{\quad}\),则
  \(\operatorname{pos}=2\) 时只能选取 \([0,6]\) 的数。
\item
  \(\rm{bool\;zero}\):当前选出的前缀是否都为 \(0\). 对于上界
  \(273967\),\(\underline{02}9753\) 也是可以取到的,这时有 \(1\)
  个前导零,不满足题目条件。这与 \(\operatorname{lim}\) 的作用类似,都用于标记特殊情况。
\end{itemize}

容易发现,若两个状态对应的四元组相同,则他们所对应的答案相同。

为了方便调试,代码使用记忆化搜索实现动态规划过程。

在记忆化数组的声明时,选择了只记录非特殊情况。因为当 \(\rm lim=true\) 或
\(\rm zero=true\) 时,状态只会出现一次,无需记录。

\subsection{代码实现}

\inputminted[frame=lines, numbers=left, fontsize=\scriptsize, tabsize=4, breaklines=true]{c++}{code/3.cpp}
