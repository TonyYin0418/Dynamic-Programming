% !TeX root = ../thuthesis-example.tex

\chapter{按数位顺序进行的动态规划}

本部分将介绍按数位顺序进行的动态规划(后文简称为数位DP)

\section{适用问题类型}

给定区间 \([L,R]\),求区间内满足一定条件的数的个数。

问题有如下特征:

\begin{itemize}
\item
  给定的条件一般与数的组成有关,比如数位之间的关系。
\item
  此类问题的数据规模一般较大。\(1\leq L<R\leq 10^{100000}\)
  在某些情况下也是可以接受的,因为是按位DP。
\item
  最终目的为计数,统计满足某一性质的数的个数。
\end{itemize}

为了方便,数位DP一般使用记忆化搜索实现。

\section{算法简述}

通常情况下,题目可转化为求区间 \([L, R]\) 内满足性质 \(p\)
的数的个数。利用前缀和的思想,先将求 \(\operatorname{ans}_{[L,R]}\)
的问题转化为
\(\operatorname{ans}_{[1,R]}-\operatorname{ans}_{[1,l-1]}\).

数位DP的核心思想,是基于:在从 \(1\)
开始逐个向更大的数遍历的过程中,过程中有很多性质重复的部分。

比如:求 \(1\sim 10000\) 中,存在某一数位为 \(7\) 的数的个数。在
\(3000\sim3999\),\(4000\sim4999\),\(5000\sim 5999\)
中,它们的后三位都是 \(\_000\sim \_ 999\),是完全相同的,因此
\(\operatorname{ans}_{[3000,3999]}=\operatorname{ans}_{[4000,4999]}=\operatorname{ans}_{[5000,5999]}\),对这些部分加以记录,相同的部分只计算一次。这与动态规划利用数组记录已经统计过的部分,相同的部分不再重复计算的性质一致。

统计答案的过程再加上记忆过程,这就是记忆化搜索,与动态规划本质无差别,因而算法名为数位
DP。

\section{应用实例}

\subsection{题目来源}

题目名称:windy 数。

题目选自:NOI2009四川省省队选拔活动。

\subsection{题目描述}

定义 \(\rm{windy}\) 数:不含前导零且相邻两个数字之差至少为 \(2\)
的正整数。

给定 \(a,b\),求 \([a,b]\) 中有多少个 \(\rm{windy}\) 数。

数据范围:\(1\leq a\leq b\leq 2\times 10^9\).

\subsection{解题思路}

设 \(\operatorname{ans}_{[a,b]}\) 代表题目所求,把
\(\operatorname{ans}_{[a,b]}\) 化为
\(\operatorname{ans}_{[a,b]}=\operatorname{ans}_{[1,b]}-\operatorname{ans}_{[1,a-1]}\).

下面求 \(\operatorname{ans}_{[1,r]}\),称 \(r\) 为上界。

定义四元组为动态规划状态:\(\rm{(pos,lst,lim,zero)}\),代表的含义分别为:

\begin{itemize}
\item
  \(\rm{int\;pos}\):当前正在尝试确定第几位,个位是第 \(1\) 位,十位是第
  \(2\) 位,以此类推。
\item
  \(\rm{int \;lst}\):本题中,需要记录上一位选的是什么数,以约束当前位置的选取。
\item
  \(\rm{bool\;lim}\):当前选出的前缀是否与上界相同,其用于约束当前数位的选取不能超过上界。比如上界为
  \(273967\),当前枚举出的前缀为 \(2739\underline{\quad}\),则
  \(\operatorname{pos}=2\) 时只能选取 \([0,6]\) 的数。
\item
  \(\rm{bool\;zero}\):当前选出的前缀是否都为 \(0\). 对于上界
  \(273967\),\(\underline{00}9000\) 也是可以取到的,这时有 \(2\)
  个前导零。这与 \(\operatorname{lim}\) 的作用类似,都用于标记特殊情况。
\end{itemize}

容易发现,只要两个状态的四元组均相同,则他们所对应的答案均相同。

为了方便调试,代码使用记忆化搜索实现动态规划过程。

在记忆化数组的声明时,选择了只记录非特殊情况。因为当 \(\rm lim=true\) 或
\(\rm zero=true\) 时,状态只会出现一次,无需记录。

\subsection{代码实现}

\inputminted[frame=lines, numbers=left, fontsize=\scriptsize, tabsize=4, breaklines=true]{c++}{code/3.cpp}

% \chapter{数学符号和公式}

% \section{数学符号}

% 中文论文的数学符号默认遵循 GB/T 3102.11—1993《物理科学和技术中使用的数学符号》
% \footnote{原 GB 3102.11—1993,自 2017 年 3 月 23 日起,该标准转为推荐性标准。}。
% 该标准参照采纳 ISO 31-11:1992 \footnote{目前已更新为 ISO 80000-2:2019。},
% 但是与 \TeX{} 默认的美国数学学会(AMS)的符号习惯有所区别。
% 具体地来说主要有以下差异:
% \begin{enumerate}
%   \item 大写希腊字母默认为斜体,如
%     \begin{equation*}
%       \Gamma \Delta \Theta \Lambda \Xi \Pi \Sigma \Upsilon \Phi \Psi \Omega.
%     \end{equation*}
%     注意有限增量符号 $\increment$ 固定使用正体,模板提供了 \cs{increment} 命令。
%   \item 小于等于号和大于等于号使用倾斜的字形 $\le$、$\ge$。
%   \item 积分号使用正体,比如 $\int$、$\oint$。
%   \item 行间公式积分号的上下限位于积分号的上下两端,比如
%     \begin{equation*}
%       \int_a^b f(x) \dif x.
%     \end{equation*}
%     行内公式为了版面的美观,统一居右侧,如 $\int_a^b f(x) \dif x$ 。
%   \item
%     偏微分符号 $\partial$ 使用正体。
%   \item
%     省略号 \cs{dots} 按照中文的习惯固定居中,比如
%     \begin{equation*}
%       1, 2, \dots, n \quad 1 + 2 + \dots + n.
%     \end{equation*}
%   \item
%     实部 $\Re$ 和虚部 $\Im$ 的字体使用罗马体。
% \end{enumerate}

% 以上数学符号样式的差异可以在模板中统一设置。
% 另外国标还有一些与 AMS 不同的符号使用习惯,需要用户在写作时进行处理:
% \begin{enumerate}
%   \item 数学常数和特殊函数名用正体,如
%     \begin{equation*}
%       \uppi = 3.14\dots; \quad
%       \symup{i}^2 = -1; \quad
%       \symup{e} = \lim_{n \to \infty} \left( 1 + \frac{1}{n} \right)^n.
%     \end{equation*}
%   \item 微分号使用正体,比如 $\dif y / \dif x$。
%   \item 向量、矩阵和张量用粗斜体(\cs{symbf}),如 $\symbf{x}$、$\symbf{\Sigma}$、$\symbfsf{T}$。
%   \item 自然对数用 $\ln x$ 不用 $\log x$。
% \end{enumerate}


% 英文论文的数学符号使用 \TeX{} 默认的样式。
% 如果有必要,也可以通过设置 \verb|math-style| 选择数学符号样式。

% 关于量和单位推荐使用
% \href{http://mirrors.ctan.org/macros/latex/contrib/siunitx/siunitx.pdf}{\pkg{siunitx}}
% 宏包,
% 可以方便地处理希腊字母以及数字与单位之间的空白,
% 比如:
% \SI{6.4e6}{m},
% \SI{9}{\micro\meter},
% \si{kg.m.s^{-1}},
% \SIrange{10}{20}{\degreeCelsius}。



% \section{数学公式}

% 数学公式可以使用 \env{equation} 和 \env{equation*} 环境。
% 注意数学公式的引用应前后带括号,建议使用 \cs{eqref} 命令,比如式\eqref{eq:example}。
% \begin{equation}
%   \frac{1}{2 \uppi \symup{i}} \int_\gamma f = \sum_{k=1}^m n(\gamma; a_k) \mathscr{R}(f; a_k)
%   \label{eq:example}
% \end{equation}
% 注意公式编号的引用应含有圆括号,可以使用 \cs{eqref} 命令。

% 多行公式尽可能在“=”处对齐,推荐使用 \env{align} 环境。
% \begin{align}
%   a & = b + c + d + e \\
%     & = f + g
% \end{align}



% \section{数学定理}

% 定理环境的格式可以使用 \pkg{amsthm} 或者 \pkg{ntheorem} 宏包配置。
% 用户在导言区载入这两者之一后,模板会自动配置 \env{thoerem}、\env{proof} 等环境。

% \begin{theorem}[Lindeberg--Lévy 中心极限定理]
%   设随机变量 $X_1, X_2, \dots, X_n$ 独立同分布, 且具有期望 $\mu$ 和有限的方差 $\sigma^2 \ne 0$,
%   记 $\bar{X}_n = \frac{1}{n} \sum_{i+1}^n X_i$,则
%   \begin{equation}
%     \lim_{n \to \infty} P \left(\frac{\sqrt{n} \left( \bar{X}_n - \mu \right)}{\sigma} \le z \right) = \Phi(z),
%   \end{equation}
%   其中 $\Phi(z)$ 是标准正态分布的分布函数。
% \end{theorem}
% \begin{proof}
%   Trivial.
% \end{proof}

% 同时模板还提供了 \env{assumption}、\env{definition}、\env{proposition}、
% \env{lemma}、\env{theorem}、\env{axiom}、\env{corollary}、\env{exercise}、
% \env{example}、\env{remar}、\env{problem}、\env{conjecture} 这些相关的环境。
