% !TeX root = ../thuthesis-example.tex

\chapter{四边形不等式优化动态规划}

\section{四边形不等式}

若对于任意整数 \(l_1,l_2,r_2,r_1\) 满足
\(l_1\le l_2\le r_2\le r_1\),若二元函数 \(w(x,y)\) 满足 \[
w(l_1,r_1)+w(l_2,r_2)\ge w(l_1,r_2)+w(l_2,r_1)\tag{1}
\] 则称 \(w\) 满足四边形不等式。四边形不等式有一个经典的等价形式: \[
w(l,r+1)+w(l+1,r)\ge w(l,r)+w(l+1,r+1)\tag{2}
\] 其中
\(l < r\)。由(1)得到(2)较为显然,由(2)推出(1)可以通过数学归纳法证得。

假设对于 \(l+x < r\) 有 \[
w(l,r+1)+w(l+x,r)\ge w(l,r)+w(l+x,r+1)\\
\] 在 \(x=1\) 时此式为(1)。

对于 \(l+x+1 < r\),根据(2)有 \[
w(l+x,r+1)+w(l+x+1,r)\ge w(l+x,r)+w(l+x+1,r+1)\\
\] 两式相加可得 \[
w(l,r+1)+w(l+x+1,r)\ge w(l,r)+w(l+x+1,r+1)
\] 同理可得 \(w(l,r+y)+w(l+x,r)\ge w(l,r)+w(l+x,r+y)\),即证。

\section{四边形不等式对一维 DP 的优化}

在优化一维 DP 时,DP 的形式往往为
\(f[i]=\min\limits_{0\le j < i}\{f[j]+w(j,i)\}\)。不妨设令 \(f[i]\)
取得最小值的 \(j\) 为 \(d[i]\),那么若 \(w\) 满足四边形不等式,则 \(d\)
单调不减,称 \(f\) 具有决策单调性。利用此性质可以使时间复杂度为
\(\mathcal{O}(n^2)\) 的计算简化为
\(\mathcal{O}(n\log n)\)。本文将不对这类优化进行细致探讨。

\section{四边形不等式对二维 DP 的优化}

在优化二维 DP 时,DP 的形式则多为
\(f[i][j]=\min\limits_{i\le k < j}\{f[i][k]+f[k+1][j]+w(i,j)\}(i < j)\),\(f[i][i]=0\)。

不妨设令 \(f[i][j]\) 取得最小值的 \(k\) 为 \(d[i][j]\),那么

\begin{itemize}

\item
  若

  \begin{enumerate}
  \def\labelenumi{\arabic{enumi}.}

  \item
    \(w\) 满足四边形不等式,
  \item
    对于任意 \(l_1\le l_2\le r_2\le r_1\),有
    \(w(l_1,r_1)\ge w(l_2,r_2)\),
  \end{enumerate}
\item
  则

  \begin{enumerate}
  \def\labelenumi{\arabic{enumi}.}

  \item
    \(f\) 也满足四边形不等式,
  \item
    \(d[i][j]\le d[i+1][j]\) 且 \(d[i][j]\le d[i][j+1]\).
  \end{enumerate}
\end{itemize}

对于满足 \(d[i][j]\le \min\{d[i+1][j],d[i][j+1]\}\) 的,我们称 \(dp\)
具有二维决策单调性。在对 \(f[i][j]\) 进行计算时,仅考察位于
\([d[i][j-1],d[i+1][j]]\) 中的 \(k\) 可以使时间复杂度为
\(\mathcal{O}(n^3)\) 的计算优化为 \(\mathcal{O}(n^2)\).

\subsection{对满足四边形不等式的证明}

关于 \(f\) 满足四边形不等式的证明,考虑使用数学归纳法。

当 \(r-l=1\) 时, \[
\begin{array}{c}
f[l][r+1]+f[l+1][r]=f[l][l+2]\ge w(l,l+2)\\
f[l][r]+f[l+1][r+1]=f[l][l+1]+f[l+1][l+2]=w(l,l+1)+w(l+1,l+2)\\
f[l][r+1]+f[l+1][r]\ge f[l][r]+f[l+1][r+1]
\end{array}
\]

设在 \(r-l < k\) 时,\(f\) 满足四边形不等式。

为方便起见,设 \(x=d[l][r+1],y=d[l+1][r]\),那么: \[
f[l][r+1]+f[l+1][r]=(f[l][x]+f[x+1][r+1]+w(l,r+1))+(f[l+1][y]+f[y+1][r]+w(l+1,r))\\
\]
\[f[l][r]+f[l+1][r+1]\le(f[l][x]+f[x+1][r]+w(l,r))+(f[l+1][y]+f[y+1][r+1]+w(l+1,r+1))
\]

欲证: \[f[l][r+1]+f[l+1][r]\ge f[l][r]+f[l+1][r+1]
\]

只需证: \[f[x+1][r+1]+f[y+1][r]+w(l,r+1)+w(l+1,r)\ge f[x+1][r]+f[y+1][r+1]+w(l,r)+w(l+1,r+1)
\]

只需证: \[
\begin{array}{c}
f[x+1][r+1]+f[y+1][r]\ge f[x+1][r]+f[y+1][r+1]\\
w(l,r+1)+w(l+1,r)\ge w(l,r)+w(l+1,r+1)
\end{array}
\]

两式分别可以由归纳假设以及 \(w\) 满足四边形不等式得到,即证 \(f\)
满足四边形不等式。

\subsection{对二维决策单调性的证明}

关于二维决策单调性的证明,设 \(x=d[i][j]\),\(i\le k < x\)。根据 \(f\)
满足四边形不等式和 \(d[i][j]\) 的最优性,有 \[
f[i][x]+f[i+1][k]\ge f[i][k]+f[i+1][x]\\
f[i][k]+f[k+1][j]\ge f[i][x]+f[x+1][j]
\] 可以得到 \[
f[i+1][k]+f[k+1][j]\ge f[i+1][x]+f[x+1][j]
\] 因此对于 \(f[i+1][j]\) 的转移,\(k\in [i,d[i][j])\) 一定不优于
\(d[i][j]\),有 \(d[i+1][j]\ge d[i][j]\)。

类似的,设 \(x < k\le j\),有 \[
f[x+1][j]+f[k+1][j-1]\ge f[x+1][j-1]+f[k+1][j]\\
f[i][k]+f[k+1][j]\ge f[i][x]+f[x+1][j]
\] 可以得到 \[
f[i][k]+f[k+1][j-1]\ge f[i][x]+f[x+1][j-1]
\] 即 \(d[i][j-1]\le d[i][j]\).

\subsection{对时间复杂度的证明}

关于时间复杂度是 \(\mathcal{O}(n^2)\) 的证明,考虑 \[
\begin{aligned}
&\sum_{i=1}^{n-1}\sum_{j=i+1}^n(d[i+1][j]-d[i][j-1])\\
=&\sum_{i=2}^n\sum_{j=i+1}^nd[i][j]-\sum_{i=1}^{n-1}\sum_{j=i}^{n-1}d[i][j]\\
=&\sum_{i=1}^{n-1}d[i][n]-\sum_{j=2}^nd[1][j]-\sum_{i=1}^nd[i][i]
\end{aligned}
\]
并且 \(d[i][j]\in[1,n]\),因此
\(\mathcal{O}(\sum_{i=1}^{n-1}\sum_{j=i+1}^n(d[i+1][j]-d[i][j-1]))=\mathcal{O}(n^2)\).

\section{应用实例一}

\subsection{题目描述}

有 \(n\) 堆直线排列的石子,第 \(i\) 堆石子有 \(a[i]\) 个。

规定每次只能合并任意相邻的两堆石子,并产生两堆石子数量之和的疲劳值。现在要将石子有序的合并成一堆,试求最小总疲劳值。

数据范围:\(1\le n\le 5000\).

\subsection{解题思路}

设 \(\operatorname{dp}[i][j]\) 为将下标在区间 \([i,j]\)
内的石子合并为一堆所需的最小疲劳值,\(w(i,j)=\sum_{k=i}^ja[k]\),那么有
\[
\operatorname{dp}[i][j]=
\left\{
\begin{array}{ll}
\min\limits_{i\le k < j}\{\operatorname{dp}[i][k]+\operatorname{dp}[k+1][j]+w(i,j)\},&i < j\\
0,&i=j\\
+\infty, &i>j
\end{array}
\right.
\] 直接转移时间复杂度为 \(\mathcal{O}(n^3)\),不能接受。发现 \(w\) 满足
\[
w(l,r+1)+w(l+1,r)=w(l,r)+w(l+1,r+1)
\] 即复合四边形不等式条件,因此可以对上述 DP 进行优化,达到
\(\mathcal{O}(n^2)\) 的时间复杂度。

\subsection{代码实现}

7.cpp

\section{应用实例二}

\subsection{题目来源}

题目名称:邮局。

题目选自:IOI2000.

\subsection{题目描述}

给定 \(n\) 个村庄在一条直线上的坐标。现在要选一些村庄建立 \(k\)
个邮局,使得每个村庄与其最近的邮局之间的距离总和最小。试求这个最小距离和。

数据范围:用 \(p_i\) 表示村庄坐标,则
\(1\le k\le 300\),\(k\le n\le 3000\),\(1 \le p_i\le 10000\).

\subsection{解题思路}

首先将村庄按其在坐标轴上的位置排序,设第 \(i\) 个村庄的坐标为 \(x_i\)。

设 \(\operatorname{dp}[i][j]\) 表示前 \(i\) 个村庄建立 \(j\)
个邮局的最小距离和,\(w(l,r)\) 表示在第 \(p\) 个村庄建立一个邮局所得到的
\(\sum_{i=l}^r|x_i-x_p|\) 的最小值。不难看出
\(p=\lfloor \frac{l+r}{2}\rfloor\),因此 \[
\begin{aligned}
w(l,r)&=(\sum_{i=p+1}^rx_i-x_p(r-p))+(x_p(p-l)-\sum_{i=l}^{p-1}x_i)\\
w(l,r)&=w(l,r-1)+x_r-x_p\\
w(l,r)&=w(l+1,r)+x_p-x_l\\
\end{aligned}
\] 可以在 DP 转移前 \(\mathcal{O}(n^2)\)
递推,也可以使用前缀和每次转移时 \(\mathcal{O}(1)\) 计算。

同时 DP 递推式有 \[
\operatorname{dp}[i][j]=
\left\{
\begin{array}{ll}
\min\limits_{1\le k\le i}\{\operatorname{dp}[k-1][j-1]+w(k,i)\},&\text{if } i\neq0,j\neq0\\
1,&\text{if }i=0 j=0\\
0,&\text{otherwise}
\end{array}
\right.
\] 直接递推是 \(\mathcal{O}(n^2k)\) 的,不能接受。发现 \[
\begin{aligned}
w(l,r)+w(l+1,r-1)&=2w(l+1,r-1)+x_r-x_l\\
w(l+1,r)+w(l,r-1)&=2w(l+1,r-1)+x_r-x_l-(x_{\lfloor \frac{l+r+1}{2}\rfloor}-x_{\lfloor \frac{l+r-1}{2}\rfloor})
\end{aligned}
\] 因为 \(x\) 递增,有
\((x_{\lfloor \frac{l+r+1}{2}\rfloor}-x_{\lfloor \frac{l+r-1}{2}\rfloor})\ge0\),所以
\(w(l,r)+w(l+1,r-1)\ge w(l+1,r)+w(l,r-1)\),\(w\)
满足四边形不等式,\(\operatorname{dp}\)
满足四边形不等式。因此对转移时计算的区间进行优化就能达到
\(\mathcal{O}(nk)\) 的时间复杂度。

\subsection{代码实现}

8.cpp
