% !TeX root = ../thuthesis-example.tex

\chapter{引言}

动态规划(Dynamic Programming, DP)与分治方法相似,都是通过组合子问题的解来求解原问题。分治方法将问题划分为
互不相交

近年来,动态规划在信息学中的应用愈发广泛。简单的动态规划模型如线性 DP、背包
DP、区间 DP
已被多数人熟练掌握并运用,本文将不再赘述。而与之形成对比的一些复杂动态规划问题由于其算法本身的难度,难以被应用或被信息学竞赛选手解答出,故本文将这些算法进行了整合并且进行了详细地解析,可以作为信息学竞赛选手及相关领域的参考资料使用。

对于一些时间复杂度本身较劣的 DP
思路,硬件运行速度的提升往往是常量级的,在数据规模快速增长时效果甚微,而直接对
DP 的状态设计进行本质优化往往具有局限性,无法系统整理并在其他 DP
上如法炮制。故本文还将介绍四种被广泛应用的 DP
转移优化方式,在保持原有状态设计和转移方式的基础上最大限度地省去不必要的计算,已达到优化时间复杂度的目的。
