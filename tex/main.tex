\section{浅析复杂动态规划模型及其优化}

【摘要】本文总结了一类以集合信息为状态、状态总数为指数级的动态规划的状态设计方法,即状态压缩法与一类有关于数位的计数类动态规划问题,即数位动态规划。对于一些时间复杂度在指数级上大于空间复杂度的动态规划,本文介绍了单调队列优化,斜率优化,四边形不等式优化及CDQ分治优化四种优化方式,可以在较大程度上使时间复杂度靠近空间复杂度。

【关键词】动态规划;状态压缩;单调队列;四边形不等式;CDQ分治

\subsection{1. 引言}

近年来,动态规划(Dynamic Programming,下文或将简称为
DP)在信息学中的应用愈发广泛。简单的动态规划模型如线性 DP、背包 DP、区间
DP
已被多数人熟练掌握并运用,本文将不再赘述。而与之形成对比的一些复杂动态规划问题由于其算法本身的难度,难以被应用或被信息学竞赛选手解答出,故本文将这些算法进行了整合并且进行了详细地解析,可以作为信息学竞赛选手及相关领域的参考资料使用。

对于一些时间复杂度本身较劣的 DP
思路,硬件运行速度的提升往往是常量级的,在数据规模快速增长时效果甚微,而直接对
DP 的状态设计进行本质优化往往具有局限性,无法系统整理并在其他 DP
上如法炮制。故本文还将介绍四种被广泛应用的 DP
转移优化方式,在保持原有状态设计和转移方式的基础上最大限度地省去不必要的计算,已达到优化时间复杂度的目的。

\subsection{2. 动态规划模型}

\subsubsection{2.1. 基于状态压缩的动态规划}

\paragraph{2.1.1. 状态压缩法}

\subparagraph{2.1.1.1. 定义}

状态压缩一般是将一个较小的集合内每个元素的状态通过特定的计算方法,转换为单个整数。这样的转换必须是双射。

在信息学竞赛中,状态的压缩和解压缩一般由二进制运算完成。究其主要原因,一方面是因为二进制可以很直接的表达一个元素集合内每个元素选中或不选中的状态,如二进制串
\(01001\) 可以表示一个大小为 \(5\) 的集合 \(S\) 中第 \(2\)、\(5\)
个元素被选中,第 \(1\)、\(3\)、\(4\)
个元素未被选中的状态。另一方面则是因为计算机在存储数据时是以二进制方式存储,直接在二进制下对数据进行处理和计算往往比其他进制要快的多。

\subparagraph{2.1.1.2. 位运算}

在二进制下对数据的处理和计算与数学中常用的加减乘除和逻辑运算略有不同,对整数在内存中的二进制位进行操作,就是位运算。下面是对几类位运算的定义解释和约定。

\begin{itemize}
\item
  与(\texttt{\&}):\(A\) 和 \(B\) 的按位与运算中,对 \(A\) 和 \(B\)
  二进制下的每一位的值进行数学中的逻辑与运算,得到结果中对应位的值。
\item
  或(\texttt{\textbar{}}):\(A\) 和 \(B\) 的按位或运算中,对 \(A\) 和
  \(B\)
  二进制下的每一位的值进行数学中的逻辑或运算,得到结果中对应位的值。
\item
  取反(\texttt{\textasciitilde{}}):\(A\) 的按位取反运算中,对 \(A\)
  二进制下的每一位的值进行数学中的逻辑非运算,得到结果中对应位的值。
\item
  异或(\texttt{\^{}}):\(A\) 和 \(B\) 的按位异或计算中,对 \(A\) 和
  \(B\) 二进制下的每一位的值进行如下计算:

  \begin{longtable}[]{@{}lll@{}}
  \toprule
  A & B & A \(\oplus\) B \\
  \midrule
  \endhead
  0 & 0 & 0 \\
  0 & 1 & 1 \\
  1 & 0 & 1 \\
  1 & 1 & 0 \\
  \bottomrule
  \end{longtable}

  得到结果中对应位的值。
\item
  \(A\) 的第 \(i\) 位表示 \(A\) 在二进制下从最低往最高位数第 \(i+1\)
  位,即 \(\frac{A}{2^i}\) 除以 \(2\) 所得余数。
\item
  左移(\texttt{\textless{}\textless{}}),右移(\texttt{\textgreater{}\textgreater{}}):\texttt{x\textless{}\textless{}y}
  表示 \(x\times 2^y\),\texttt{x} 的第 \(i\) 位为
  \texttt{(x\textless{}\textless{}y)} 的第 \(i+y\)
  位,\texttt{(x\textless{}\textless{}y)} 的 \(0\sim (y-1)\) 位均为
  \(0\)。\texttt{x\textgreater{}\textgreater{}y} 表示
  \(\frac{x}{2^y}\),\texttt{x} 的第 \(i\) 位(\(i\ge y\))为
  \texttt{(x\textgreater{}\textgreater{}y)} 的第 \(i-y\) 位。
\end{itemize}

设全集为 \(\Omega\),\(|\Omega|=n\),用 \(n\) 位二进制整数 \(x\) 描述
\(\Omega\) 的一个子集 \(S\),那么 \(x\) 的第 \(i-1\) 位为 \(1\) 则表示
\(\Omega\) 中第 \(i\) 个元素属于 \(S\),为 \(0\) 则表示 \(\Omega\) 中第
\(i\) 个元素不属于
\(S\)。基于此,对于数学中对集合之间的运算和关系的命题,位运算也有对应计算方式。

\begin{itemize}
\tightlist
\item
  \(A\) 与 \(B\) 交集:\texttt{A\&B}。对于表示第 \(i+1\)
  个元素是否在交集中的第 \(i\) 位,该位值为 \(1\) 当且仅当 \(A\) 的第
  \(i\) 位为 \(1\) 且 \(B\) 的第 \(i\) 位为 \(1\),即 \(A\) 包含第
  \(i+1\) 个元素且 \(B\) 包含第 \(i+1\) 个元素。
\item
  \(A\) 与 \(B\) 并集:\texttt{A\textbar{}B}。对于表示第 \(i+1\)
  个元素是否在并集中的第 \(i\) 位,该位值为 \(1\) 当且仅当 \(A\) 的第
  \(i\) 位为 \(1\) 或 \(B\) 的第 \(i\) 位为 \(1\),即 \(A\) 包含第
  \(i+1\) 个元素或 \(B\) 包含第 \(i+1\) 个元素。
\item
  \(A\) 的补集:设 \texttt{O=((1\textless{}\textless{}n)-1)},这样有
  \texttt{O} 的 \(0\) 到 \(n-1\) 位均为 \(1\),则 \(A\)
  的补集为\texttt{O\^{}A}。对于表示第 \(i+1\) 个元素是否在补集中的第
  \(i\) 位,因为 \(O\) 第 \(i\) 位为 \(1\),因此该位为 \(1\) 当且仅当
  \(A\) 的第 \(i\) 位为 \(0\),即 \(A\) 不包含第 \(i+1\) 个元素。
\item
  \(A/B\):\texttt{A\^{}(A\&B)}。对于表示第 \(i+1\) 个元素是否在 \(A/B\)
  中的第 \(i\) 位,该位值为 \(1\) 当且仅当 \(A\) 第 \(i\) 位为 \(1\) 且
  \(A\cap B\) 第 \(i\) 位为 \(0\)。
\item
  \(p:\) \(A\) 是 \(B\)
  的子集:\texttt{(A\textbar{}B)==B}。若该表达式为真,那么对于第 \(i\)
  位,若 \(B\) 第 \(i\) 位为 \(0\),则 \(A\) 的第 \(i\) 位为 \(0\),若
  \(B\) 第 \(i\) 位为 \(1\),则 \(A\) 的第 \(i\) 位可取 \(0/1\)。即第
  \(i\) 个元素若不属于 \(B\),则第 \(i\) 个元素也不属于 \(A\),命题
  \(p\) 为真。
\end{itemize}

\paragraph{2.1.2. 基于状态压缩的动态规划类型}

为满足 DP
的无后效性,存储对应值的索引往往需要包括可以描述该阶段状态的全部信息。而对于一类需要记录整个集合内每个元素状态的问题,朴素的对某一数组指针进行中括号运算在程序中略显乏力,设计的状态往往会冗长且在转移时拖沓,因此状态压缩优化状态就成为了更优的选择。

以一类经典的 NP
问题,旅行商问题(TSP)为例,设计传统的动态规划状态需要描述每一个点是否被到达过,因此描述需要
\(n\) 维,其中 \(n\) 为点数。\(n\) 若不为常数,以动态的维数来实现 DP
的转移本身就较为困难,而就算假设 \(n\) 为常数,例如 \(6\),状态也需写成
\(f(0/1,0/1,0/1,0/1,0/1,0/1)\),转移时枚举转移前状态和转移后状态更是需要
\(7\)
个循环以上。而采取状态压缩,将每一个点是否被到达过表示为映射到的整数第
\(i\) 位是否为 \(1\),则状态只需写成 \(f(S)\),转移最少只需 \(2\)
个循环。

\subparagraph{应用实例1 {[}SCOI2005{]} 互不侵犯}

题目描述

在国际象棋中,国王能攻击它上、下、左、右、左上、左下、右上、右下八个方向上附近各一个格子,共
\(8\) 个。

现在在 \(n\times n\) 的国际象棋棋盘上摆放 \(k\)
个国王,使他们互不攻击,试求共有多少种摆放方案。

\(1\le n\le 9,0\le k\le n\times n\)

解题思路

定义 \(f[i][j][S]\) 表示从第 \(1\) 行摆放到第 \(i\) 行,已经摆放了 \(j\)
个国王,第 \(i\) 行每个格子摆放状态为 \(S\) 的合法摆放方案数。其中 \(S\)
的第 \(i\) 位为 \(1\) 则表示此行从左往右数第 \(i+1\)
个格子摆放了国王,反之则没有摆放国王。

判断单行状态 \(S\) 单独出现是否合法,考虑判断
\texttt{(S\&(S\textless{}\textless{}1))==0}
是否为真。若为真则代表对于任意
\(i\),\texttt{S\&(S\textless{}\textless{}1)} 的第 \(i\) 位均为
\(0\),意味着 \(S\) 第 \(i\) 位为 \(1\) 和 \(S\) 的第 \(i-1\) 位为 \(1\)
不同时成立,即没有在同一行摆放左右相邻的国王。

判断两个状态 \(S_1\)、\(S_2\) 是否可以作为合法的相邻两行出现,考虑判断
\texttt{((S1\&S2)\textbar{}\textbar{}(S1\&(S2\textless{}\textless{}1))\textbar{}\textbar{}(S1\&(S2\textgreater{}\textgreater{}1)))==0}
是否为真。假设为假,那么如下三条至少存在一条为真

\begin{itemize}
\tightlist
\item
  \texttt{(S1\&S2)} 存在一位 \(i\) 使得其值为 \(1\),则 \(S_1\) 第 \(i\)
  位为 \(1\) 且 \(S_2\) 第 \(i\) 位为 \(1\),即上下相邻,不合法;
\item
  \texttt{(S1\&(S2\textless{}\textless{}1))} 存在一位 \(i\) 使得其值为
  \(1\),则 \(S_1\) 第 \(i\) 位为 \(1\) 且 \(S_2\) 第 \(i-1\) 位为
  \(1\),即互为左下和右上的关系,不合法;
\item
  \texttt{(S1\&(S2\textgreater{}\textgreater{}1))} 存在一位 \(i\)
  使得其值为 \(1\),则 \(S_1\) 第 \(i\) 位为 \(1\) 且 \(S_2\) 第 \(i+1\)
  位为 \(1\),即互为左上和右下的关系,不合法。
\end{itemize}

这三条判断和单行是否合法的判断覆盖了所有可能的不合法情况,提供了
\(O(1)\) 的计算方法。而 C++ 中 \texttt{\_\_builtin\_popcount(x)}
函数可以以近似 \(O(1)\) 的效率计算 \(x\) 二进制中 \(1\)
的个数,以下简写为 \texttt{pc(x)}。

转移考虑从小到大枚举 \(i\in[1,n]\),枚举 \(j\in[0,k]\),枚举合法单行状态
\(S_1\in[0,2^n)\)、\(S_2\in[0,2^n)\)。在 \(S_1\) 和 \(S_2\)
可以作为合法的相邻两行出现时,有转移 \[
f[i][j+pc(S_2)][S_2]:=f[i][j+pc(S_2)][S_2]+f[i-1][j][S_1]
\] 初始状态为
\(f[0][0][0]=1\),即没有填任何行时,仅在没有摆放任何国王且该行为空时存在一个方案。最终答案为
\[
Ans=\sum_{S\in[0,2^n)}f[n][k][S]
\] 即摆放了所有行,一共摆放了 \(k\)
个国王,最后一行为任意状态的合法方案总和。整体时间复杂度为
\(O(nkA+2^n)\),空间复杂度为 \(O(nk2^n)\),其中 \(A\) 为合法单行
\(S_1\)、\(S_2\) 组成的不同合法相邻行个数,在 \(n\le 9\) 时满足
\(A\le 683\)。

代码实现

\begin{Shaded}
\begin{Highlighting}[]
\PreprocessorTok{\#include}\ImportTok{\textless{}bits/stdc++.h\textgreater{}}
\KeywordTok{using} \KeywordTok{namespace}\NormalTok{ std}\OperatorTok{;}
\DataTypeTok{long} \DataTypeTok{long}\NormalTok{ n}\OperatorTok{,}\NormalTok{k}\OperatorTok{,}\NormalTok{f}\OperatorTok{[}\DecValTok{15}\OperatorTok{][}\DecValTok{105}\OperatorTok{][(}\DecValTok{1}\OperatorTok{\textless{}\textless{}}\DecValTok{9}\OperatorTok{)];}
\PreprocessorTok{\#define pc }\FunctionTok{\_\_builtin\_popcount}
\DataTypeTok{int}\NormalTok{ main}\OperatorTok{()\{}
\NormalTok{    cin}\OperatorTok{\textgreater{}\textgreater{}}\NormalTok{n}\OperatorTok{\textgreater{}\textgreater{}}\NormalTok{k}\OperatorTok{;}
\NormalTok{    f}\OperatorTok{[}\DecValTok{0}\OperatorTok{][}\DecValTok{0}\OperatorTok{][}\DecValTok{0}\OperatorTok{]=}\DecValTok{1}\OperatorTok{;}
    \ControlFlowTok{for}\OperatorTok{(}\DataTypeTok{int}\NormalTok{ i}\OperatorTok{=}\DecValTok{1}\OperatorTok{;}\NormalTok{i}\OperatorTok{\textless{}=}\NormalTok{n}\OperatorTok{;}\NormalTok{i}\OperatorTok{++)\{}
        \ControlFlowTok{for}\OperatorTok{(}\DataTypeTok{int}\NormalTok{ S1}\OperatorTok{=}\DecValTok{0}\OperatorTok{;}\NormalTok{S1}\OperatorTok{\textless{}(}\DecValTok{1}\OperatorTok{\textless{}\textless{}}\NormalTok{n}\OperatorTok{);}\NormalTok{S1}\OperatorTok{++)\{}
            \ControlFlowTok{if}\OperatorTok{((}\NormalTok{S1}\OperatorTok{\&(}\NormalTok{S1}\OperatorTok{\textless{}\textless{}}\DecValTok{1}\OperatorTok{))==}\DecValTok{0}\OperatorTok{)\{}
                \ControlFlowTok{for}\OperatorTok{(}\DataTypeTok{int}\NormalTok{ S2}\OperatorTok{=}\DecValTok{0}\OperatorTok{;}\NormalTok{S2}\OperatorTok{\textless{}(}\DecValTok{1}\OperatorTok{\textless{}\textless{}}\NormalTok{n}\OperatorTok{);}\NormalTok{S2}\OperatorTok{++)\{}
                    \ControlFlowTok{if}\OperatorTok{(((}\NormalTok{S2}\OperatorTok{\&(}\NormalTok{S2}\OperatorTok{\textless{}\textless{}}\DecValTok{1}\OperatorTok{))==}\DecValTok{0}\OperatorTok{)\&\&(((}\NormalTok{S1}\OperatorTok{\&}\NormalTok{S2}\OperatorTok{)||(}\NormalTok{S1}\OperatorTok{\&(}\NormalTok{S2}\OperatorTok{\textless{}\textless{}}\DecValTok{1}\OperatorTok{))||(}\NormalTok{S1}\OperatorTok{\&(}\NormalTok{S2}\OperatorTok{\textgreater{}\textgreater{}}\DecValTok{1}\OperatorTok{)))==}\DecValTok{0}\OperatorTok{))\{}
                        \ControlFlowTok{for}\OperatorTok{(}\DataTypeTok{int}\NormalTok{ j}\OperatorTok{=}\DecValTok{0}\OperatorTok{;}\NormalTok{j}\OperatorTok{\textless{}=}\NormalTok{k}\OperatorTok{;}\NormalTok{j}\OperatorTok{++)\{}
\NormalTok{                            f}\OperatorTok{[}\NormalTok{i}\OperatorTok{][}\NormalTok{j}\OperatorTok{+}\NormalTok{pc}\OperatorTok{(}\NormalTok{S2}\OperatorTok{)][}\NormalTok{S2}\OperatorTok{]+=}\NormalTok{f}\OperatorTok{[}\NormalTok{i}\OperatorTok{{-}}\DecValTok{1}\OperatorTok{][}\NormalTok{j}\OperatorTok{][}\NormalTok{S1}\OperatorTok{];}
                        \OperatorTok{\}}
                    \OperatorTok{\}}
                \OperatorTok{\}}
            \OperatorTok{\}}
        \OperatorTok{\}}
    \OperatorTok{\}}
    \DataTypeTok{long} \DataTypeTok{long}\NormalTok{ ans}\OperatorTok{=}\DecValTok{0}\OperatorTok{;}
    \ControlFlowTok{for}\OperatorTok{(}\DataTypeTok{int}\NormalTok{ S}\OperatorTok{=}\DecValTok{0}\OperatorTok{;}\NormalTok{S}\OperatorTok{\textless{}(}\DecValTok{1}\OperatorTok{\textless{}\textless{}}\NormalTok{n}\OperatorTok{);}\NormalTok{S}\OperatorTok{++)}\NormalTok{ans}\OperatorTok{+=}\NormalTok{f}\OperatorTok{[}\NormalTok{n}\OperatorTok{][}\NormalTok{k}\OperatorTok{][}\NormalTok{S}\OperatorTok{];}
\NormalTok{    cout}\OperatorTok{\textless{}\textless{}}\NormalTok{ans}\OperatorTok{\textless{}\textless{}}\NormalTok{endl}\OperatorTok{;}
    \ControlFlowTok{return} \DecValTok{0}\OperatorTok{;}
\OperatorTok{\}}
\end{Highlighting}
\end{Shaded}

\subparagraph{应用实例2 {[}NOI2015{]} 寿司晚宴}

题目描述

给定 \(n\),设全集为
\(\Omega=\{x\mid x\in[2,n]\}\)。求有多少个不同的二元组 \((A,B)\),满足:

\begin{itemize}
\tightlist
\item
  \(A,B\) 为 \(\Omega\) 的子集;
\item
  对于任意 \(x\in A,y\in B\),有 \(\gcd(x,y)=1\)。
\end{itemize}

答案对 \(p\) 取模。

\(2\le n\le 500\)。

解题思路

显然不能对 \(500\) 以内的 \(95\)
个质因数进行状态压缩和转移,因为这样时空复杂度至少是 \(O(2^{95})\)
的,远超一般计算机每秒算力。

但可以发现对于任意 \(x\in[2,500]\),\(x\) 分解质因数后包含的 \(\gt19\)
的质因子不会超过 \(1\) 个,\(\le19\) 的质因数仅 \(8\)
个,可以状态压缩。而我们只关注 \(\Omega\)
中每个数分解质因数后的质因子集合,因此一个数 \(i\) 可以被表示为二元组
\((p_i,S_i)\),其中 \(p_i\) 表示 \(i\) 分解质因数后 \(\gt19\)
的质因子,没有则为 \(1\);\(S_i\) 表示 \(i\) 分解质因数后包含的
\(\le19\) 的质因子的集合,\(S_i\) 共 \(8\) 位,第 \(x\) 位为 \(1\)
则表示从小到大第 \(x+1\) 个质数是 \(i\) 的因子,反之则不是。

按二元组第一位给 \(n-1\) 个数分组后,可以看出除 \(p_i=1\)
的组,其余组不能即有元素属于 \(A\)、又有元素属于 \(B\)。且将 \(A\)
集合中的所有元素二元组第二位按位或后,\(B\) 集合对应值与之进行按位与应为
\(0\)。

因此可以对整体设计状态,对 \(p_i=1\) 的组合 \(p_i\gt1\)
的组分别进行转移。

设 \(f[a][b]\) 表示 \(A\) 集合中所有元素第二位按位或的值为 \(a\),\(B\)
集合中所有元素第二位按位或的值为 \(b\) 且没有任意一个元素同时属于
\(A,B\),没有任意一组中即有元素属于 \(A\)、又有元素属于 \(B\) 的方案数。

\begin{itemize}
\item
  对于 \(p_i=1\) 的组,有 \[
  \begin{aligned}
  f'[a|S_i][b]&:=f[a][b]\\
  f'[a][b|S_i]&:=f[a][b]\\
  \end{aligned}
  \] 其中 \(f'\) 为新的 \(f\) 数组,在枚举完 \(a,b\) 后替换 \(f\)。即
  \(f\) 数组不是边枚举 \(a,b\) 边更新的。下同。
\item
  对于每一组,设 \(dp1[a][b]\) 为当前组没有任一元素在 \(B\) 中的方案数,
  \(dp2[a][b]\) 为当前组没有任一元素在 \(A\) 中的方案数,有 \[
  \begin{aligned}
  dp1'[a|S_i][b]&:=dp1[a][b]\\
  dp2'[a][b|S_i]&:=dp2[a][b]\\
  \end{aligned}
  \] 在这一组更新初,将 \(dp1,dp2\) 的值赋为
  \(f\)。在更新完这一组所有元素后,有 \[
  f'[a][b]=dp1[a][b]+dp2[a][b]-f[a][b]
  \] 因为这一组即没有元素在 \(A\) 中,也没有元素在 \(B\) 中会被算重。
\end{itemize}

最后有 \[
Ans=\sum_{S_1\cap S_2=\varnothing}f[S_1][S_2]
\] 整体时间复杂度为 \(O(n2^{16})\),空间复杂度为 \(O(2^{16})\)。

代码实现

\begin{Shaded}
\begin{Highlighting}[]
\PreprocessorTok{\#include }\ImportTok{\textless{}bits/stdc++.h\textgreater{}}
\KeywordTok{using} \KeywordTok{namespace}\NormalTok{ std}\OperatorTok{;}
\KeywordTok{typedef} \DataTypeTok{long} \DataTypeTok{long}\NormalTok{ ll}\OperatorTok{;}
\AttributeTok{const} \DataTypeTok{int}\NormalTok{ maxn}\OperatorTok{=}\DecValTok{505}\OperatorTok{,}\NormalTok{maxs}\OperatorTok{=(}\DecValTok{1}\OperatorTok{\textless{}\textless{}}\DecValTok{8}\OperatorTok{),}\NormalTok{prn}\OperatorTok{=}\DecValTok{8}\OperatorTok{,}\NormalTok{pr}\OperatorTok{[}\NormalTok{prn}\OperatorTok{]=\{}\DecValTok{2}\OperatorTok{,}\DecValTok{3}\OperatorTok{,}\DecValTok{5}\OperatorTok{,}\DecValTok{7}\OperatorTok{,}\DecValTok{11}\OperatorTok{,}\DecValTok{13}\OperatorTok{,}\DecValTok{17}\OperatorTok{,}\DecValTok{19}\OperatorTok{\};}
\KeywordTok{inline} \DataTypeTok{int}\NormalTok{ read}\OperatorTok{()\{}
    \DataTypeTok{int}\NormalTok{ x}\OperatorTok{=}\DecValTok{0}\OperatorTok{,}\NormalTok{f}\OperatorTok{=}\DecValTok{1}\OperatorTok{,}\NormalTok{ch}\OperatorTok{=}\NormalTok{getchar}\OperatorTok{();}
    \ControlFlowTok{while}\OperatorTok{(}\NormalTok{ch}\OperatorTok{\textless{}}\CharTok{\textquotesingle{}0\textquotesingle{}}\OperatorTok{||}\NormalTok{ch}\OperatorTok{\textgreater{}}\CharTok{\textquotesingle{}9\textquotesingle{}}\OperatorTok{)\{}\ControlFlowTok{if}\OperatorTok{(}\NormalTok{ch}\OperatorTok{==}\CharTok{\textquotesingle{}{-}\textquotesingle{}}\OperatorTok{)}\NormalTok{f}\OperatorTok{={-}}\DecValTok{1}\OperatorTok{;}\NormalTok{ch}\OperatorTok{=}\NormalTok{getchar}\OperatorTok{();\}}
    \ControlFlowTok{while}\OperatorTok{(}\NormalTok{ch}\OperatorTok{\textgreater{}=}\CharTok{\textquotesingle{}0\textquotesingle{}}\OperatorTok{\&\&}\NormalTok{ch}\OperatorTok{\textless{}=}\CharTok{\textquotesingle{}9\textquotesingle{}}\OperatorTok{)\{}\NormalTok{x}\OperatorTok{=(}\NormalTok{x}\OperatorTok{\textless{}\textless{}}\DecValTok{3}\OperatorTok{)+(}\NormalTok{x}\OperatorTok{\textless{}\textless{}}\DecValTok{1}\OperatorTok{)+}\NormalTok{ch}\OperatorTok{{-}}\DecValTok{48}\OperatorTok{;}\NormalTok{ch}\OperatorTok{=}\NormalTok{getchar}\OperatorTok{();\}}
    \ControlFlowTok{return}\NormalTok{ x}\OperatorTok{*}\NormalTok{f}\OperatorTok{;}
\OperatorTok{\}}
\DataTypeTok{int}\NormalTok{ n}\OperatorTok{,}\NormalTok{mod}\OperatorTok{;}
\KeywordTok{struct}\NormalTok{ num}\OperatorTok{\{}\DataTypeTok{int}\NormalTok{ v}\OperatorTok{,}\NormalTok{key}\OperatorTok{,}\NormalTok{S}\OperatorTok{;\}}\NormalTok{a}\OperatorTok{[}\NormalTok{maxn}\OperatorTok{];}
\DataTypeTok{bool}\NormalTok{ cmp}\OperatorTok{(}\NormalTok{num a}\OperatorTok{,}\NormalTok{num b}\OperatorTok{)\{}\ControlFlowTok{return}\NormalTok{ a}\OperatorTok{.}\NormalTok{key}\OperatorTok{==}\NormalTok{b}\OperatorTok{.}\NormalTok{key}\OperatorTok{?}\NormalTok{a}\OperatorTok{.}\NormalTok{v}\OperatorTok{\textless{}}\NormalTok{b}\OperatorTok{.}\NormalTok{v}\OperatorTok{:}\NormalTok{a}\OperatorTok{.}\NormalTok{key}\OperatorTok{\textless{}}\NormalTok{b}\OperatorTok{.}\NormalTok{key}\OperatorTok{;\}}
\DataTypeTok{int}\NormalTok{ f}\OperatorTok{[}\NormalTok{maxs}\OperatorTok{][}\NormalTok{maxs}\OperatorTok{],}\NormalTok{dp1}\OperatorTok{[}\NormalTok{maxs}\OperatorTok{][}\NormalTok{maxs}\OperatorTok{],}\NormalTok{dp2}\OperatorTok{[}\NormalTok{maxs}\OperatorTok{][}\NormalTok{maxs}\OperatorTok{],}\NormalTok{\_dp1}\OperatorTok{[}\NormalTok{maxs}\OperatorTok{][}\NormalTok{maxs}\OperatorTok{],}\NormalTok{\_dp2}\OperatorTok{[}\NormalTok{maxs}\OperatorTok{][}\NormalTok{maxs}\OperatorTok{];}
\KeywordTok{inline} \DataTypeTok{void}\NormalTok{ upd}\OperatorTok{(}\DataTypeTok{int} \OperatorTok{\&}\NormalTok{x}\OperatorTok{,}\DataTypeTok{int}\NormalTok{ y}\OperatorTok{)\{}\NormalTok{x}\OperatorTok{=}\NormalTok{x}\OperatorTok{+}\NormalTok{y}\OperatorTok{;}\ControlFlowTok{if}\OperatorTok{(}\NormalTok{x}\OperatorTok{\textgreater{}=}\NormalTok{mod}\OperatorTok{)}\NormalTok{x}\OperatorTok{{-}=}\NormalTok{mod}\OperatorTok{;\}}
\DataTypeTok{int}\NormalTok{ main}\OperatorTok{()\{}
\NormalTok{    n}\OperatorTok{=}\NormalTok{read}\OperatorTok{();}\NormalTok{mod}\OperatorTok{=}\NormalTok{read}\OperatorTok{();}
    \ControlFlowTok{for}\OperatorTok{(}\DataTypeTok{int}\NormalTok{ i}\OperatorTok{=}\DecValTok{1}\OperatorTok{;}\NormalTok{i}\OperatorTok{\textless{}}\NormalTok{n}\OperatorTok{;}\NormalTok{i}\OperatorTok{++)\{}
\NormalTok{        a}\OperatorTok{[}\NormalTok{i}\OperatorTok{].}\NormalTok{v}\OperatorTok{=}\NormalTok{a}\OperatorTok{[}\NormalTok{i}\OperatorTok{].}\NormalTok{key}\OperatorTok{=}\NormalTok{i}\OperatorTok{+}\DecValTok{1}\OperatorTok{;}
        \ControlFlowTok{for}\OperatorTok{(}\DataTypeTok{int}\NormalTok{ j}\OperatorTok{=}\DecValTok{0}\OperatorTok{;}\NormalTok{j}\OperatorTok{\textless{}}\NormalTok{prn}\OperatorTok{;}\NormalTok{j}\OperatorTok{++)\{}
            \ControlFlowTok{if}\OperatorTok{(!(}\NormalTok{a}\OperatorTok{[}\NormalTok{i}\OperatorTok{].}\NormalTok{key}\OperatorTok{\%}\NormalTok{pr}\OperatorTok{[}\NormalTok{j}\OperatorTok{]))}\NormalTok{a}\OperatorTok{[}\NormalTok{i}\OperatorTok{].}\NormalTok{S}\OperatorTok{|=(}\DecValTok{1}\OperatorTok{\textless{}\textless{}}\NormalTok{j}\OperatorTok{);}
            \ControlFlowTok{while}\OperatorTok{(!(}\NormalTok{a}\OperatorTok{[}\NormalTok{i}\OperatorTok{].}\NormalTok{key}\OperatorTok{\%}\NormalTok{pr}\OperatorTok{[}\NormalTok{j}\OperatorTok{]))}\NormalTok{a}\OperatorTok{[}\NormalTok{i}\OperatorTok{].}\NormalTok{key}\OperatorTok{/=}\NormalTok{pr}\OperatorTok{[}\NormalTok{j}\OperatorTok{];}
        \OperatorTok{\}}
    \OperatorTok{\}}
\NormalTok{    sort}\OperatorTok{(}\NormalTok{a}\OperatorTok{+}\DecValTok{1}\OperatorTok{,}\NormalTok{a}\OperatorTok{+}\NormalTok{n}\OperatorTok{,}\NormalTok{cmp}\OperatorTok{);}
\NormalTok{    f}\OperatorTok{[}\DecValTok{0}\OperatorTok{][}\DecValTok{0}\OperatorTok{]=}\DecValTok{1}\OperatorTok{;}
    \DataTypeTok{int}\NormalTok{ st}\OperatorTok{=}\DecValTok{1}\OperatorTok{;}
    \ControlFlowTok{while}\OperatorTok{(}\NormalTok{a}\OperatorTok{[}\NormalTok{st}\OperatorTok{].}\NormalTok{key}\OperatorTok{==}\DecValTok{1}\OperatorTok{)\{}
\NormalTok{        memset}\OperatorTok{(}\NormalTok{\_dp1}\OperatorTok{,}\DecValTok{0}\OperatorTok{,}\KeywordTok{sizeof}\OperatorTok{(}\NormalTok{\_dp1}\OperatorTok{));}
        \ControlFlowTok{for}\OperatorTok{(}\DataTypeTok{int}\NormalTok{ s1}\OperatorTok{=}\DecValTok{0}\OperatorTok{;}\NormalTok{s1}\OperatorTok{\textless{}(}\DecValTok{1}\OperatorTok{\textless{}\textless{}}\NormalTok{prn}\OperatorTok{);}\NormalTok{s1}\OperatorTok{++)}
        \ControlFlowTok{for}\OperatorTok{(}\DataTypeTok{int}\NormalTok{ s2}\OperatorTok{=}\DecValTok{0}\OperatorTok{;}\NormalTok{s2}\OperatorTok{\textless{}(}\DecValTok{1}\OperatorTok{\textless{}\textless{}}\NormalTok{prn}\OperatorTok{);}\NormalTok{s2}\OperatorTok{++)\{}
\NormalTok{            upd}\OperatorTok{(}\NormalTok{\_dp1}\OperatorTok{[}\NormalTok{s1}\OperatorTok{|}\NormalTok{a}\OperatorTok{[}\NormalTok{st}\OperatorTok{].}\NormalTok{S}\OperatorTok{][}\NormalTok{s2}\OperatorTok{],}\NormalTok{f}\OperatorTok{[}\NormalTok{s1}\OperatorTok{][}\NormalTok{s2}\OperatorTok{]);}
\NormalTok{            upd}\OperatorTok{(}\NormalTok{\_dp1}\OperatorTok{[}\NormalTok{s1}\OperatorTok{][}\NormalTok{s2}\OperatorTok{|}\NormalTok{a}\OperatorTok{[}\NormalTok{st}\OperatorTok{].}\NormalTok{S}\OperatorTok{],}\NormalTok{f}\OperatorTok{[}\NormalTok{s1}\OperatorTok{][}\NormalTok{s2}\OperatorTok{]);}
        \OperatorTok{\}}
        \ControlFlowTok{for}\OperatorTok{(}\DataTypeTok{int}\NormalTok{ s1}\OperatorTok{=}\DecValTok{0}\OperatorTok{;}\NormalTok{s1}\OperatorTok{\textless{}(}\DecValTok{1}\OperatorTok{\textless{}\textless{}}\NormalTok{prn}\OperatorTok{);}\NormalTok{s1}\OperatorTok{++)}
        \ControlFlowTok{for}\OperatorTok{(}\DataTypeTok{int}\NormalTok{ s2}\OperatorTok{=}\DecValTok{0}\OperatorTok{;}\NormalTok{s2}\OperatorTok{\textless{}(}\DecValTok{1}\OperatorTok{\textless{}\textless{}}\NormalTok{prn}\OperatorTok{);}\NormalTok{s2}\OperatorTok{++)}
\NormalTok{            upd}\OperatorTok{(}\NormalTok{f}\OperatorTok{[}\NormalTok{s1}\OperatorTok{][}\NormalTok{s2}\OperatorTok{],}\NormalTok{\_dp1}\OperatorTok{[}\NormalTok{s1}\OperatorTok{][}\NormalTok{s2}\OperatorTok{]);}
\NormalTok{        st}\OperatorTok{++;}
    \OperatorTok{\}}
    \ControlFlowTok{for}\OperatorTok{(}\DataTypeTok{int}\NormalTok{ i}\OperatorTok{=}\NormalTok{st}\OperatorTok{;}\NormalTok{i}\OperatorTok{\textless{}}\NormalTok{n}\OperatorTok{;}\NormalTok{i}\OperatorTok{++)\{}
        \ControlFlowTok{if}\OperatorTok{(}\NormalTok{a}\OperatorTok{[}\NormalTok{i}\OperatorTok{].}\NormalTok{key}\OperatorTok{!=}\NormalTok{a}\OperatorTok{[}\NormalTok{i}\OperatorTok{{-}}\DecValTok{1}\OperatorTok{].}\NormalTok{key}\OperatorTok{)}
        \ControlFlowTok{for}\OperatorTok{(}\DataTypeTok{int}\NormalTok{ s1}\OperatorTok{=}\DecValTok{0}\OperatorTok{;}\NormalTok{s1}\OperatorTok{\textless{}(}\DecValTok{1}\OperatorTok{\textless{}\textless{}}\NormalTok{prn}\OperatorTok{);}\NormalTok{s1}\OperatorTok{++)}
        \ControlFlowTok{for}\OperatorTok{(}\DataTypeTok{int}\NormalTok{ s2}\OperatorTok{=}\DecValTok{0}\OperatorTok{;}\NormalTok{s2}\OperatorTok{\textless{}(}\DecValTok{1}\OperatorTok{\textless{}\textless{}}\NormalTok{prn}\OperatorTok{);}\NormalTok{s2}\OperatorTok{++)}
\NormalTok{            dp1}\OperatorTok{[}\NormalTok{s1}\OperatorTok{][}\NormalTok{s2}\OperatorTok{]=}\NormalTok{dp2}\OperatorTok{[}\NormalTok{s1}\OperatorTok{][}\NormalTok{s2}\OperatorTok{]=}\NormalTok{f}\OperatorTok{[}\NormalTok{s1}\OperatorTok{][}\NormalTok{s2}\OperatorTok{];}
\NormalTok{        memset}\OperatorTok{(}\NormalTok{\_dp1}\OperatorTok{,}\DecValTok{0}\OperatorTok{,}\KeywordTok{sizeof}\OperatorTok{(}\NormalTok{\_dp1}\OperatorTok{));}
\NormalTok{        memset}\OperatorTok{(}\NormalTok{\_dp2}\OperatorTok{,}\DecValTok{0}\OperatorTok{,}\KeywordTok{sizeof}\OperatorTok{(}\NormalTok{\_dp2}\OperatorTok{));}
        \ControlFlowTok{for}\OperatorTok{(}\DataTypeTok{int}\NormalTok{ s1}\OperatorTok{=}\DecValTok{0}\OperatorTok{;}\NormalTok{s1}\OperatorTok{\textless{}(}\DecValTok{1}\OperatorTok{\textless{}\textless{}}\NormalTok{prn}\OperatorTok{);}\NormalTok{s1}\OperatorTok{++)}
        \ControlFlowTok{for}\OperatorTok{(}\DataTypeTok{int}\NormalTok{ s2}\OperatorTok{=}\DecValTok{0}\OperatorTok{;}\NormalTok{s2}\OperatorTok{\textless{}(}\DecValTok{1}\OperatorTok{\textless{}\textless{}}\NormalTok{prn}\OperatorTok{);}\NormalTok{s2}\OperatorTok{++)\{}
\NormalTok{            upd}\OperatorTok{(}\NormalTok{\_dp1}\OperatorTok{[}\NormalTok{s1}\OperatorTok{|}\NormalTok{a}\OperatorTok{[}\NormalTok{i}\OperatorTok{].}\NormalTok{S}\OperatorTok{][}\NormalTok{s2}\OperatorTok{],}\NormalTok{dp1}\OperatorTok{[}\NormalTok{s1}\OperatorTok{][}\NormalTok{s2}\OperatorTok{]);}
\NormalTok{            upd}\OperatorTok{(}\NormalTok{\_dp2}\OperatorTok{[}\NormalTok{s1}\OperatorTok{][}\NormalTok{s2}\OperatorTok{|}\NormalTok{a}\OperatorTok{[}\NormalTok{i}\OperatorTok{].}\NormalTok{S}\OperatorTok{],}\NormalTok{dp2}\OperatorTok{[}\NormalTok{s1}\OperatorTok{][}\NormalTok{s2}\OperatorTok{]);}
        \OperatorTok{\}}
        \ControlFlowTok{for}\OperatorTok{(}\DataTypeTok{int}\NormalTok{ s1}\OperatorTok{=}\DecValTok{0}\OperatorTok{;}\NormalTok{s1}\OperatorTok{\textless{}(}\DecValTok{1}\OperatorTok{\textless{}\textless{}}\NormalTok{prn}\OperatorTok{);}\NormalTok{s1}\OperatorTok{++)}
        \ControlFlowTok{for}\OperatorTok{(}\DataTypeTok{int}\NormalTok{ s2}\OperatorTok{=}\DecValTok{0}\OperatorTok{;}\NormalTok{s2}\OperatorTok{\textless{}(}\DecValTok{1}\OperatorTok{\textless{}\textless{}}\NormalTok{prn}\OperatorTok{);}\NormalTok{s2}\OperatorTok{++)}
\NormalTok{            upd}\OperatorTok{(}\NormalTok{dp1}\OperatorTok{[}\NormalTok{s1}\OperatorTok{][}\NormalTok{s2}\OperatorTok{],}\NormalTok{\_dp1}\OperatorTok{[}\NormalTok{s1}\OperatorTok{][}\NormalTok{s2}\OperatorTok{]),}
\NormalTok{            upd}\OperatorTok{(}\NormalTok{dp2}\OperatorTok{[}\NormalTok{s1}\OperatorTok{][}\NormalTok{s2}\OperatorTok{],}\NormalTok{\_dp2}\OperatorTok{[}\NormalTok{s1}\OperatorTok{][}\NormalTok{s2}\OperatorTok{]);}
        \ControlFlowTok{if}\OperatorTok{(}\NormalTok{a}\OperatorTok{[}\NormalTok{i}\OperatorTok{].}\NormalTok{key}\OperatorTok{!=}\NormalTok{a}\OperatorTok{[}\NormalTok{i}\OperatorTok{+}\DecValTok{1}\OperatorTok{].}\NormalTok{key}\OperatorTok{)}
        \ControlFlowTok{for}\OperatorTok{(}\DataTypeTok{int}\NormalTok{ s1}\OperatorTok{=}\DecValTok{0}\OperatorTok{;}\NormalTok{s1}\OperatorTok{\textless{}(}\DecValTok{1}\OperatorTok{\textless{}\textless{}}\NormalTok{prn}\OperatorTok{);}\NormalTok{s1}\OperatorTok{++)}
        \ControlFlowTok{for}\OperatorTok{(}\DataTypeTok{int}\NormalTok{ s2}\OperatorTok{=}\DecValTok{0}\OperatorTok{;}\NormalTok{s2}\OperatorTok{\textless{}(}\DecValTok{1}\OperatorTok{\textless{}\textless{}}\NormalTok{prn}\OperatorTok{);}\NormalTok{s2}\OperatorTok{++)}
\NormalTok{            f}\OperatorTok{[}\NormalTok{s1}\OperatorTok{][}\NormalTok{s2}\OperatorTok{]=((}\NormalTok{dp1}\OperatorTok{[}\NormalTok{s1}\OperatorTok{][}\NormalTok{s2}\OperatorTok{]+}\NormalTok{dp2}\OperatorTok{[}\NormalTok{s1}\OperatorTok{][}\NormalTok{s2}\OperatorTok{])\%}\NormalTok{mod}\OperatorTok{{-}}\NormalTok{f}\OperatorTok{[}\NormalTok{s1}\OperatorTok{][}\NormalTok{s2}\OperatorTok{]+}\NormalTok{mod}\OperatorTok{)\%}\NormalTok{mod}\OperatorTok{;}
    \OperatorTok{\}}
    \DataTypeTok{int}\NormalTok{ ans}\OperatorTok{=}\DecValTok{0}\OperatorTok{;}
    \ControlFlowTok{for}\OperatorTok{(}\DataTypeTok{int}\NormalTok{ s1}\OperatorTok{=}\DecValTok{0}\OperatorTok{;}\NormalTok{s1}\OperatorTok{\textless{}(}\DecValTok{1}\OperatorTok{\textless{}\textless{}}\NormalTok{prn}\OperatorTok{);}\NormalTok{s1}\OperatorTok{++)}
    \ControlFlowTok{for}\OperatorTok{(}\DataTypeTok{int}\NormalTok{ s2}\OperatorTok{=}\DecValTok{0}\OperatorTok{;}\NormalTok{s2}\OperatorTok{\textless{}(}\DecValTok{1}\OperatorTok{\textless{}\textless{}}\NormalTok{prn}\OperatorTok{);}\NormalTok{s2}\OperatorTok{++)}
        \ControlFlowTok{if}\OperatorTok{((}\NormalTok{s1}\OperatorTok{\&}\NormalTok{s2}\OperatorTok{)==}\DecValTok{0}\OperatorTok{)}\NormalTok{upd}\OperatorTok{(}\NormalTok{ans}\OperatorTok{,}\NormalTok{f}\OperatorTok{[}\NormalTok{s1}\OperatorTok{][}\NormalTok{s2}\OperatorTok{]);}
\NormalTok{    cout}\OperatorTok{\textless{}\textless{}}\NormalTok{ans}\OperatorTok{\textless{}\textless{}}\CharTok{\textquotesingle{}}\SpecialCharTok{\textbackslash{}n}\CharTok{\textquotesingle{}}\OperatorTok{;}
    \ControlFlowTok{return} \DecValTok{0}\OperatorTok{;}
\OperatorTok{\}}
\end{Highlighting}
\end{Shaded}

\subsubsection{2.2. 数位动态规划}

\paragraph{2.2.1. 数位动态规划类型}

给定区间 \([L,R]\),求区间内满足一定条件的数的个数。

问题有如下特征:

\begin{itemize}
\item
  给定的条件一般与数的组成有关,比如数位之间的关系。
\item
  此类问题的数据规模一般较大。\(1\leq L<R\leq 10^{100000}\)
  在某些情况下也是可以接受的,因为是按位DP。
\item
  最终目的为计数,统计满足某一性质的数的个数。
\end{itemize}

为了方便,数位DP一般使用\textbf{记忆化搜索}实现。

\paragraph{2.2.2. 算法简述}

通常情况下,题目可转化为求区间 \([L, R]\) 内满足性质 \(p\)
的数的个数。利用前缀和的思想,先将求 \(\operatorname{ans}_{[L,R]}\)
的问题转化为
\(\operatorname{ans}_{[1,R]}-\operatorname{ans}_{[1,l-1]}\).

数位DP的核心思想,是基于:在从 \(1\)
开始逐个向更大的数遍历的过程中,过程中有很多性质重复的部分。

比如:求 \(1\sim 10000\) 中,存在某一数位为 \(7\) 的数的个数。在
\(3000\sim3999\),\(4000\sim4999\),\(5000\sim 5999\)
中,它们的后三位都是 \(\_000\sim \_ 999\),是完全相同的,因此
\(\operatorname{ans}_{[3000,3999]}=\operatorname{ans}_{[4000,4999]}=\operatorname{ans}_{[5000,5999]}\),对这些部分加以记录,相同的部分只计算一次。这与动态规划利用数组记录已经统计过的部分,相同的部分不再重复计算的性质一致。

统计答案的过程再加上记忆过程,这就是记忆化搜索,与动态规划本质无差别,因而算法名为数位
DP。

\paragraph{应用实例1 {[}SCOI2009{]} windy 数}

\subparagraph{题目描述}

定义 \(\rm{windy}\) 数:\textbf{不含前导零且相邻两个数字之差至少为 \(2\)
的正整数}。

给定 \(a,b\),求 \([a,b]\) 中有多少个 \(\rm{windy}\) 数。

\(1\leq a\leq b\leq 2\times 10^9\).

\subparagraph{解题思路}

设 \(\operatorname{ans}_{[a,b]}\) 代表题目所求,把
\(\operatorname{ans}_{[a,b]}\) 化为
\(\operatorname{ans}_{[a,b]}=\operatorname{ans}_{[1,b]}-\operatorname{ans}_{[1,a-1]}\).

下面求 \(\operatorname{ans}_{[1,r]}\),称 \(r\) 为上界。

定义四元组为动态规划状态:\(\rm{(pos,lst,lim,zero)}\),代表的含义分别为:

\begin{itemize}
\tightlist
\item
  \(\rm{int\;pos}\):当前正在尝试确定第几位,个位是第 \(1\) 位,十位是第
  \(2\) 位,以此类推。
\item
  \(\rm{int \;lst}\):本题中,需要记录上一位选的是什么数,以约束当前位置的选取。
\item
  \(\rm{bool\;lim}\):当前选出的前缀是否与上界相同,其用于约束当前数位的选取不能超过上界。比如上界为
  \(273967\),当前枚举出的前缀为 \(2739\underline{\quad}\),则
  \(\operatorname{pos}=2\) 时只能选取 \([0,6]\) 的数。
\item
  \(\rm{bool\;zero}\):当前选出的前缀是否都为 \(0\). 对于上界
  \(273967\),\(\underline{00}9000\) 也是可以取到的,这时有 \(2\)
  个前导零。这与 \(\operatorname{lim}\) 的作用类似,都用于标记特殊情况。
\end{itemize}

容易发现,只要两个状态的四元组均相同,则他们所对应的答案均相同。

为了方便调试,代码使用记忆化搜索实现动态规划过程。

在记忆化数组的声明时,选择了只记录非特殊情况。因为当 \(\rm lim=true\) 或
\(\rm zero=true\) 时,状态只会出现一次,无需记录。

\subparagraph{代码实现}

\begin{Shaded}
\begin{Highlighting}[]
\PreprocessorTok{\#include }\ImportTok{\textless{}bits/stdc++.h\textgreater{}}
\KeywordTok{using} \KeywordTok{namespace}\NormalTok{ std}\OperatorTok{;}
\AttributeTok{const} \DataTypeTok{int}\NormalTok{ MAXN }\OperatorTok{=} \DecValTok{20}\OperatorTok{;}
\DataTypeTok{int}\NormalTok{ a}\OperatorTok{,}\NormalTok{ b}\OperatorTok{,}\NormalTok{ num}\OperatorTok{[}\NormalTok{MAXN}\OperatorTok{];}
\DataTypeTok{int}\NormalTok{ dp}\OperatorTok{[}\NormalTok{MAXN}\OperatorTok{][}\NormalTok{MAXN}\OperatorTok{];}
\DataTypeTok{int}\NormalTok{ dfs}\OperatorTok{(}\DataTypeTok{int}\NormalTok{ pos}\OperatorTok{,} \DataTypeTok{int}\NormalTok{ lst}\OperatorTok{,} \DataTypeTok{bool}\NormalTok{ lim}\OperatorTok{,} \DataTypeTok{bool}\NormalTok{ zero}\OperatorTok{)} \OperatorTok{\{}
    \ControlFlowTok{if}\OperatorTok{(}\NormalTok{pos }\OperatorTok{==} \DecValTok{0}\OperatorTok{)} \ControlFlowTok{return} \DecValTok{1}\OperatorTok{;}
    \ControlFlowTok{if}\OperatorTok{(!}\NormalTok{lim }\OperatorTok{\&\&} \OperatorTok{!}\NormalTok{zero }\OperatorTok{\&\&}\NormalTok{ dp}\OperatorTok{[}\NormalTok{pos}\OperatorTok{][}\NormalTok{lst}\OperatorTok{]} \OperatorTok{!=} \OperatorTok{{-}}\DecValTok{1}\OperatorTok{)} \ControlFlowTok{return}\NormalTok{ dp}\OperatorTok{[}\NormalTok{pos}\OperatorTok{][}\NormalTok{lst}\OperatorTok{];}
    \DataTypeTok{int}\NormalTok{ ret }\OperatorTok{=} \DecValTok{0}\OperatorTok{;}
    \DataTypeTok{int}\NormalTok{ mx }\OperatorTok{=}\NormalTok{ lim }\OperatorTok{?}\NormalTok{ num}\OperatorTok{[}\NormalTok{pos}\OperatorTok{]} \OperatorTok{:} \DecValTok{9}\OperatorTok{;}
    \ControlFlowTok{for}\OperatorTok{(}\DataTypeTok{int}\NormalTok{ i }\OperatorTok{=} \DecValTok{0}\OperatorTok{;}\NormalTok{ i }\OperatorTok{\textless{}=}\NormalTok{ mx}\OperatorTok{;}\NormalTok{ i}\OperatorTok{++)} \ControlFlowTok{if}\OperatorTok{(}\NormalTok{abs}\OperatorTok{(}\NormalTok{i }\OperatorTok{{-}}\NormalTok{ lst}\OperatorTok{)} \OperatorTok{\textgreater{}=} \DecValTok{2}\OperatorTok{)}
\NormalTok{        ret }\OperatorTok{+=}\NormalTok{ dfs}\OperatorTok{(}\NormalTok{pos }\OperatorTok{{-}} \DecValTok{1}\OperatorTok{,} \OperatorTok{(}\NormalTok{zero }\OperatorTok{\&} \OperatorTok{(}\NormalTok{i }\OperatorTok{==} \DecValTok{0}\OperatorTok{))} \OperatorTok{?} \OperatorTok{{-}}\DecValTok{2} \OperatorTok{:}\NormalTok{ i}\OperatorTok{,}
\NormalTok{                   lim }\OperatorTok{\&} \OperatorTok{(}\NormalTok{i }\OperatorTok{==}\NormalTok{ num}\OperatorTok{[}\NormalTok{pos}\OperatorTok{]),}\NormalTok{ zero }\OperatorTok{\&} \OperatorTok{(}\NormalTok{i }\OperatorTok{==} \DecValTok{0}\OperatorTok{));}
    \ControlFlowTok{if}\OperatorTok{(!}\NormalTok{lim }\OperatorTok{\&\&} \OperatorTok{!}\NormalTok{zero}\OperatorTok{)}\NormalTok{ dp}\OperatorTok{[}\NormalTok{pos}\OperatorTok{][}\NormalTok{lst}\OperatorTok{]} \OperatorTok{=}\NormalTok{ ret}\OperatorTok{;}
    \ControlFlowTok{return}\NormalTok{ ret}\OperatorTok{;}
\OperatorTok{\}}
\DataTypeTok{int}\NormalTok{ solve}\OperatorTok{(}\DataTypeTok{int}\NormalTok{ x}\OperatorTok{)} \OperatorTok{\{}
    \DataTypeTok{int}\NormalTok{ p }\OperatorTok{=} \DecValTok{0}\OperatorTok{;}
    \ControlFlowTok{while}\OperatorTok{(}\NormalTok{x}\OperatorTok{)}
\NormalTok{        num}\OperatorTok{[++}\NormalTok{p}\OperatorTok{]} \OperatorTok{=}\NormalTok{ x }\OperatorTok{\%} \DecValTok{10}\OperatorTok{,}\NormalTok{ x }\OperatorTok{/=} \DecValTok{10}\OperatorTok{;}
\NormalTok{    memset}\OperatorTok{(}\NormalTok{dp}\OperatorTok{,} \OperatorTok{{-}}\DecValTok{1}\OperatorTok{,} \KeywordTok{sizeof}\OperatorTok{(}\NormalTok{dp}\OperatorTok{));}
    \ControlFlowTok{return}\NormalTok{ dfs}\OperatorTok{(}\NormalTok{p}\OperatorTok{,} \OperatorTok{{-}}\DecValTok{2}\OperatorTok{,} \DecValTok{1}\OperatorTok{,} \DecValTok{1}\OperatorTok{);}
\OperatorTok{\}}
\DataTypeTok{int}\NormalTok{ main}\OperatorTok{()} \OperatorTok{\{}
\NormalTok{    cin }\OperatorTok{\textgreater{}\textgreater{}}\NormalTok{ a }\OperatorTok{\textgreater{}\textgreater{}}\NormalTok{ b}\OperatorTok{;}
\NormalTok{    cout }\OperatorTok{\textless{}\textless{}}\NormalTok{ solve}\OperatorTok{(}\NormalTok{b}\OperatorTok{)} \OperatorTok{{-}}\NormalTok{ solve}\OperatorTok{(}\NormalTok{a }\OperatorTok{{-}} \DecValTok{1}\OperatorTok{)} \OperatorTok{\textless{}\textless{}}\NormalTok{ endl}\OperatorTok{;}
    \ControlFlowTok{return} \DecValTok{0}\OperatorTok{;}
\OperatorTok{\}}
\end{Highlighting}
\end{Shaded}

\subsection{3. 动态规划优化}

\subsubsection{3.1. 单调队列优化 DP}

\paragraph{3.1.1. 单调队列}

队列是一种先进先出的数据结构,其队尾可加入元素,队首可取出或弹出元素。单调队列则是在队列的基础上,让队尾可以弹出元素以使队列中从队首到队尾相邻元素均满足定义的偏序关系。在信息学竞赛中,这样的元素多为二元组或三元组,其中一维为数组下标。

如经典问题``滑动窗口''中,题目给定一长度为 \(n\) 的数组 \(a\),定义
\(s_i=\min_{j\in(\max(1,i-k),i]} a_j\),要求出对于任意 \(i\in[1,n],s_i\)
的值。

较为常用的解决区间最值问题的数据结构,如线段树、树状数组、ST
表等,均需要 \(O(n\log n)\) 的时间复杂度。

而使用单调队列,则可做到 \(O(n)\)
的时间复杂度和空间复杂度。具体而言,维护二元组 \((a_i,i)\)
使队列中元素满足严格偏序关系
\(\lt=\{((x_1,y_1),(x_2,y_2))\mid x_1\lt x_2\and y_1\lt y_2\}\),下标
\(i\) 从小到大遍历 \([1,n]\) 时,加入元素
\((a_i,i)\)。若加入元素前队列非空且队尾元素不满足偏序关系,则一直弹出队尾元素直到队列为空或队尾元素满足偏序关系为止。加入元素
\((a_i,i)\) 后,在队首元素第二维 \(id\) 不满足 \(id\gt i-k\)
时弹出队首元素直至满足条件。因为第二维加入时单调递增,第一维在队列中也单调递增,所以队列中最小值在队首。而在弹出队首操作中满足了队列中所有元素均在
\((i-k,i]\) 中,且先前在队列中且第二维属于 \((i-k,i]\)
的不会因为队首弹出操作在此之前弹出,因此队首元素 \((a_{h},h)\)
的第一维即为 \(s_i\)。

\paragraph{3.1.2. 适用 DP 类型}

使用单调队列优化 DP
时,往往是题目中有一些条件使得在没有该条件时的最优决策不合法了,因此使用单调队列排除不可能在接下来成为最优决策点的元素,保留其余元素。

\subparagraph{应用实例1 {[}POI2014{]} PTA-Little Bird}

题目描述

给定 \(n\) 颗树,第 \(i\) 颗树的高度为 \(h_i\),有一只鸟要从第 \(1\)
颗树飞到第 \(n\) 颗树,它的初始劳累值为 \(0\)。

如果这只鸟当前在第 \(i\) 颗树,那么它接下来可以飞到
\(i+1,i+2,\dots,i+k\) 颗树。

如果它从一颗高度为 \(h_i\) 的树飞到高度为 \(h_j\) 的树,且
\(h_i\le h_j\),那么它的劳累值会 \(+1\),反之若 \(h_i\gt h_j\)
那么劳累值不变。

有 \(Q\) 次询问,每次询问指定 \(k\),求这只鸟从第 \(1\) 颗树飞到第 \(n\)
颗树的劳累值最少是多少。

\(2\le n\le 10^6\),\(1\le Q\le 25\)

解题思路

设 \(dp[i]\) 表示从第 \(1\) 颗树飞到第 \(i\)
颗树所需的最小劳累值,那么转移有 \[
dp[i]=\min_{j\in[\max(1,i-k),i]}(dp[j]+[h[j]\le h[i]])
\] 其中 \([p]\) 表示若命题 \(p\) 为真则值为 \(1\),否则为
\(0\)。这样直接转移是 \(O(Qn^2)\) 的,考虑优化。

设 \(m_i=\min_{j\in[\max(1,i-k),i]}dp[j]\),不难看出有 \[
\min_{j\in[\max(1,i-k),i]\and dp[j]=m_i}(dp[j]+[h[j]\le h[i]])\le\min_{j\in[\max(1,i-k),i]\and dp[j]\gt m_i}(dp[j]+[h[j]\le h[i]])
\] 因为 \[
\text{左式}\le\min_{j\in[\max(1,i-k),i]\and dp[j]=m_i}(dp[j]+1)\le m_i+1
\]

\[
\text{右式}\ge\min_{j\in[\max(1,i-k),i]\and dp[j]\gt m_i}(dp[j])\ge m_i+1
\]

即证。

因此考虑在每次询问时用类似``滑动窗口''的方式,\(O(n)\) 维护
\([\max(1,i-k),i]\) 中 \(dp\) 最小值所对应下标。考虑到要取最小值,还要
\(+[h[j]\le h[i]]\),对下标递增,\(dp\) 值相同的元素,在队列中需要使其
\(h\) 值递减。这样取到的队首元素就是合法转移点中,\(dp\) 值最小且 \(h\)
最大的元素。

整体时间复杂度为 \(O(qn)\),空间复杂度为 \(O(n)\)。

代码实现

\begin{Shaded}
\begin{Highlighting}[]
\PreprocessorTok{\#include }\ImportTok{\textless{}bits/stdc++.h\textgreater{}}
\KeywordTok{using} \KeywordTok{namespace}\NormalTok{ std}\OperatorTok{;}
\KeywordTok{typedef} \DataTypeTok{long} \DataTypeTok{long}\NormalTok{ ll}\OperatorTok{;}
\AttributeTok{const} \DataTypeTok{int}\NormalTok{ maxn}\OperatorTok{=}\FloatTok{1e6}\OperatorTok{+}\DecValTok{5}\OperatorTok{;}
\DataTypeTok{int}\NormalTok{ n}\OperatorTok{,}\NormalTok{Q}\OperatorTok{,}\NormalTok{k}\OperatorTok{,}\NormalTok{h}\OperatorTok{[}\NormalTok{maxn}\OperatorTok{],}\NormalTok{dp}\OperatorTok{[}\NormalTok{maxn}\OperatorTok{];}
\DataTypeTok{int}\NormalTok{ q}\OperatorTok{[}\NormalTok{maxn}\OperatorTok{],}\NormalTok{hd}\OperatorTok{,}\NormalTok{tl}\OperatorTok{;}
\DataTypeTok{int}\NormalTok{ main}\OperatorTok{()\{}
\NormalTok{    ios}\OperatorTok{::}\NormalTok{sync\_with\_stdio}\OperatorTok{(}\DecValTok{0}\OperatorTok{);}
\NormalTok{    cin}\OperatorTok{\textgreater{}\textgreater{}}\NormalTok{n}\OperatorTok{;}
    \ControlFlowTok{for}\OperatorTok{(}\DataTypeTok{int}\NormalTok{ i}\OperatorTok{=}\DecValTok{1}\OperatorTok{;}\NormalTok{i}\OperatorTok{\textless{}=}\NormalTok{n}\OperatorTok{;}\NormalTok{i}\OperatorTok{++)}\NormalTok{cin}\OperatorTok{\textgreater{}\textgreater{}}\NormalTok{h}\OperatorTok{[}\NormalTok{i}\OperatorTok{];}
\NormalTok{    cin}\OperatorTok{\textgreater{}\textgreater{}}\NormalTok{Q}\OperatorTok{;}
    \ControlFlowTok{while}\OperatorTok{(}\NormalTok{Q}\OperatorTok{{-}{-})\{}
\NormalTok{        cin}\OperatorTok{\textgreater{}\textgreater{}}\NormalTok{k}\OperatorTok{;}
\NormalTok{        q}\OperatorTok{[}\NormalTok{hd}\OperatorTok{=}\NormalTok{tl}\OperatorTok{=}\DecValTok{1}\OperatorTok{]=}\DecValTok{1}\OperatorTok{;}
        \ControlFlowTok{for}\OperatorTok{(}\DataTypeTok{int}\NormalTok{ i}\OperatorTok{=}\DecValTok{2}\OperatorTok{;}\NormalTok{i}\OperatorTok{\textless{}=}\NormalTok{n}\OperatorTok{;}\NormalTok{i}\OperatorTok{++)\{}
            \ControlFlowTok{while}\OperatorTok{(}\NormalTok{hd}\OperatorTok{\textless{}=}\NormalTok{tl}\OperatorTok{\&\&}\NormalTok{i}\OperatorTok{{-}}\NormalTok{q}\OperatorTok{[}\NormalTok{hd}\OperatorTok{]\textgreater{}}\NormalTok{k}\OperatorTok{)}\NormalTok{hd}\OperatorTok{++;}
\NormalTok{            dp}\OperatorTok{[}\NormalTok{i}\OperatorTok{]=}\NormalTok{dp}\OperatorTok{[}\NormalTok{q}\OperatorTok{[}\NormalTok{hd}\OperatorTok{]]+(}\NormalTok{h}\OperatorTok{[}\NormalTok{q}\OperatorTok{[}\NormalTok{hd}\OperatorTok{]]\textless{}=}\NormalTok{h}\OperatorTok{[}\NormalTok{i}\OperatorTok{]);}
            \ControlFlowTok{while}\OperatorTok{(}\NormalTok{hd}\OperatorTok{\textless{}=}\NormalTok{tl}\OperatorTok{\&\&(}\NormalTok{dp}\OperatorTok{[}\NormalTok{q}\OperatorTok{[}\NormalTok{tl}\OperatorTok{]]\textgreater{}}\NormalTok{dp}\OperatorTok{[}\NormalTok{i}\OperatorTok{]||(}\NormalTok{dp}\OperatorTok{[}\NormalTok{q}\OperatorTok{[}\NormalTok{tl}\OperatorTok{]]==}\NormalTok{dp}\OperatorTok{[}\NormalTok{i}\OperatorTok{]\&\&}\NormalTok{h}\OperatorTok{[}\NormalTok{q}\OperatorTok{[}\NormalTok{tl}\OperatorTok{]]\textless{}=}\NormalTok{h}\OperatorTok{[}\NormalTok{i}\OperatorTok{])))}\NormalTok{tl}\OperatorTok{{-}{-};}
\NormalTok{            q}\OperatorTok{[++}\NormalTok{tl}\OperatorTok{]=}\NormalTok{i}\OperatorTok{;}
        \OperatorTok{\}}
\NormalTok{        printf}\OperatorTok{(}\StringTok{"}\SpecialCharTok{\%d\textbackslash{}n}\StringTok{"}\OperatorTok{,}\NormalTok{dp}\OperatorTok{[}\NormalTok{n}\OperatorTok{]);}
    \OperatorTok{\}}
    \ControlFlowTok{return} \DecValTok{0}\OperatorTok{;}
\OperatorTok{\}}
\end{Highlighting}
\end{Shaded}

\subparagraph{应用实例2 {[}USACO13NOV{]} Pogo-Cow S}

题目描述

给定 \(n\) 个目标点,每个目标点在数轴上有一个坐标 \(x_i\)
和一个可获得分数 \(p_i\)。

定义一个长为 \(|p|\) 的序列 \(p\) 合法当且仅当 \(x_{p_i}\) 关于 \(i\)
单调递增或单调递减,且
\(\forall i\in[1,|p|-2],|x_{p_i}-x_{p_{i+1}}|\le |x_{p_{i+1}}-x_{p_{i+2}}|\),该序列权值为
\(\sum_{i\in[1,|p|]}w_{p_i}\)。

试求出所有合法序列的权值的最大值。

\(1\le n\le 10^3\)

解题思路

单调递减只需令所有 \(x_i:=-x_i\)
就可以归约到单调递增的情况,因此下文将仅讨论序列中 \(x_{p_i}\)
单调递增的情况。

首先将目标点按 \(x\) 排序,记录 \(dp[i][\Delta x]\) 表示满足最后一个值为
\(i\) 且倒数第二个值与最后一个值对应目标点 \(x\) 差值为 \(\Delta x\)
的序列中序列权值的最大值,那么应当有 \[
dp[i][x_i-x_j]=\max_{j\in[1,i)\and x'\le x_i-x_j}(dp[j][x'])+p_i
\] 但就算是离散地考虑 \(x'\) 一维,这样转移也是 \(O(n^3)\)
的,考虑优化。

发现若对于同一 \(i\),有 \(a\le b\) 且
\(dp[i][a]\ge dp[i][b]\),那么忽略 \(b\) 不会对任何 \(dp[i][x]\)
有影响。而对同一 \(j\),在 \(i\) 单调递增的情况下,限制产生贡献的
\(dp[j][x]\) 的 \(x\) 值单调不减。因此考虑对每一个 \(i\)
建立一个单调队列,每个元素 \((\Delta x,dp[i][\Delta x])\)
两维均单调递增。在队列内元素个数大于
\(1\),且从队首数第二个元素第一维满足限制时,弹出队首元素。这样能够找到可能对
\(dp[i][x_i-x_j]\) 产生贡献的最大的满足限制 \(x'\le x_i-x_j\) 的
\(x'\),对应 \(dp[i][x']\) 也是满足限制下的最大值。

这样一共需要 \(O(n)\) 个单调队列,每个队列中元素至多 \(O(n)\)
个,转移整体时间复杂度为 \(O(n^2)\),空间复杂度为 \(O(n^2)\)。

代码实现

\begin{Shaded}
\begin{Highlighting}[]
\PreprocessorTok{\#include }\ImportTok{\textless{}bits/stdc++.h\textgreater{}}
\KeywordTok{using} \KeywordTok{namespace}\NormalTok{ std}\OperatorTok{;}
\KeywordTok{typedef} \DataTypeTok{long} \DataTypeTok{long}\NormalTok{ ll}\OperatorTok{;}
\AttributeTok{const} \DataTypeTok{int}\NormalTok{ maxn}\OperatorTok{=}\FloatTok{1e3}\OperatorTok{+}\DecValTok{5}\OperatorTok{;}
\DataTypeTok{int}\NormalTok{ n}\OperatorTok{;}
\KeywordTok{struct}\NormalTok{ node}\OperatorTok{\{}\DataTypeTok{int}\NormalTok{ x}\OperatorTok{,}\NormalTok{w}\OperatorTok{;\}}\NormalTok{a}\OperatorTok{[}\NormalTok{maxn}\OperatorTok{];}
\DataTypeTok{bool}\NormalTok{ cmp}\OperatorTok{(}\NormalTok{node a}\OperatorTok{,}\NormalTok{node b}\OperatorTok{)\{}\ControlFlowTok{return}\NormalTok{ a}\OperatorTok{.}\NormalTok{x}\OperatorTok{\textless{}}\NormalTok{b}\OperatorTok{.}\NormalTok{x}\OperatorTok{;\}}
\KeywordTok{struct}\NormalTok{ Que}\OperatorTok{\{}
\NormalTok{    node v}\OperatorTok{[}\NormalTok{maxn}\OperatorTok{];}
    \DataTypeTok{int}\NormalTok{ tail}\OperatorTok{,}\NormalTok{head}\OperatorTok{;}
    \KeywordTok{inline} \DataTypeTok{void}\NormalTok{ push}\OperatorTok{(}\NormalTok{node x}\OperatorTok{)\{}
        \ControlFlowTok{while}\OperatorTok{(}\NormalTok{tail}\OperatorTok{\&\&}\NormalTok{v}\OperatorTok{[}\NormalTok{tail}\OperatorTok{].}\NormalTok{x}\OperatorTok{==}\NormalTok{x}\OperatorTok{.}\NormalTok{x}\OperatorTok{\&\&}\NormalTok{v}\OperatorTok{[}\NormalTok{tail}\OperatorTok{].}\NormalTok{w}\OperatorTok{\textless{}=}\NormalTok{x}\OperatorTok{.}\NormalTok{w}\OperatorTok{)}\NormalTok{tail}\OperatorTok{{-}{-};}
        \ControlFlowTok{if}\OperatorTok{(}\NormalTok{x}\OperatorTok{.}\NormalTok{w}\OperatorTok{\textgreater{}}\NormalTok{v}\OperatorTok{[}\NormalTok{tail}\OperatorTok{].}\NormalTok{w}\OperatorTok{||!}\NormalTok{tail}\OperatorTok{)}\NormalTok{v}\OperatorTok{[++}\NormalTok{tail}\OperatorTok{]=}\NormalTok{x}\OperatorTok{;}
    \OperatorTok{\}}
    \KeywordTok{inline} \DataTypeTok{void}\NormalTok{ pop}\OperatorTok{(}\DataTypeTok{int}\NormalTok{ x}\OperatorTok{)\{}
        \ControlFlowTok{while}\OperatorTok{(}\NormalTok{head}\OperatorTok{\textless{}}\NormalTok{tail}\OperatorTok{\&\&}\NormalTok{x}\OperatorTok{\textgreater{}=}\NormalTok{v}\OperatorTok{[}\NormalTok{head}\OperatorTok{+}\DecValTok{1}\OperatorTok{].}\NormalTok{x}\OperatorTok{)}\NormalTok{head}\OperatorTok{++;}
    \OperatorTok{\}}
    \KeywordTok{inline} \DataTypeTok{void}\NormalTok{ clear}\OperatorTok{()\{}
\NormalTok{        tail}\OperatorTok{=}\NormalTok{head}\OperatorTok{=}\DecValTok{0}\OperatorTok{;}
    \OperatorTok{\}}
\OperatorTok{\}}\NormalTok{s}\OperatorTok{[}\NormalTok{maxn}\OperatorTok{];}
\DataTypeTok{int}\NormalTok{ main}\OperatorTok{()\{}
\NormalTok{    ios}\OperatorTok{::}\NormalTok{sync\_with\_stdio}\OperatorTok{(}\DecValTok{0}\OperatorTok{);}
\NormalTok{    cin}\OperatorTok{\textgreater{}\textgreater{}}\NormalTok{n}\OperatorTok{;}
    \ControlFlowTok{for}\OperatorTok{(}\DataTypeTok{int}\NormalTok{ i}\OperatorTok{=}\DecValTok{1}\OperatorTok{;}\NormalTok{i}\OperatorTok{\textless{}=}\NormalTok{n}\OperatorTok{;}\NormalTok{i}\OperatorTok{++)}\NormalTok{cin}\OperatorTok{\textgreater{}\textgreater{}}\NormalTok{a}\OperatorTok{[}\NormalTok{i}\OperatorTok{].}\NormalTok{x}\OperatorTok{\textgreater{}\textgreater{}}\NormalTok{a}\OperatorTok{[}\NormalTok{i}\OperatorTok{].}\NormalTok{w}\OperatorTok{;}
\NormalTok{    sort}\OperatorTok{(}\NormalTok{a}\OperatorTok{+}\DecValTok{1}\OperatorTok{,}\NormalTok{a}\OperatorTok{+}\DecValTok{1}\OperatorTok{+}\NormalTok{n}\OperatorTok{,}\NormalTok{cmp}\OperatorTok{);}
    \DataTypeTok{int}\NormalTok{ ans}\OperatorTok{=}\DecValTok{0}\OperatorTok{;}
    \ControlFlowTok{for}\OperatorTok{(}\DataTypeTok{int}\NormalTok{ i}\OperatorTok{=}\DecValTok{1}\OperatorTok{;}\NormalTok{i}\OperatorTok{\textless{}=}\NormalTok{n}\OperatorTok{;}\NormalTok{i}\OperatorTok{++)\{}
\NormalTok{        s}\OperatorTok{[}\NormalTok{i}\OperatorTok{].}\NormalTok{push}\OperatorTok{((}\NormalTok{node}\OperatorTok{)\{{-}}\DecValTok{1}\OperatorTok{,}\NormalTok{a}\OperatorTok{[}\NormalTok{i}\OperatorTok{].}\NormalTok{w}\OperatorTok{\});}
        \ControlFlowTok{for}\OperatorTok{(}\DataTypeTok{int}\NormalTok{ j}\OperatorTok{=}\NormalTok{i}\OperatorTok{{-}}\DecValTok{1}\OperatorTok{;}\NormalTok{j}\OperatorTok{\textgreater{}=}\DecValTok{1}\OperatorTok{;}\NormalTok{j}\OperatorTok{{-}{-})\{}
\NormalTok{            s}\OperatorTok{[}\NormalTok{j}\OperatorTok{].}\NormalTok{pop}\OperatorTok{(}\NormalTok{a}\OperatorTok{[}\NormalTok{i}\OperatorTok{].}\NormalTok{x}\OperatorTok{{-}}\NormalTok{a}\OperatorTok{[}\NormalTok{j}\OperatorTok{].}\NormalTok{x}\OperatorTok{);}
\NormalTok{            s}\OperatorTok{[}\NormalTok{i}\OperatorTok{].}\NormalTok{push}\OperatorTok{((}\NormalTok{node}\OperatorTok{)\{}\NormalTok{a}\OperatorTok{[}\NormalTok{i}\OperatorTok{].}\NormalTok{x}\OperatorTok{{-}}\NormalTok{a}\OperatorTok{[}\NormalTok{j}\OperatorTok{].}\NormalTok{x}\OperatorTok{,}\NormalTok{s}\OperatorTok{[}\NormalTok{j}\OperatorTok{].}\NormalTok{v}\OperatorTok{[}\NormalTok{s}\OperatorTok{[}\NormalTok{j}\OperatorTok{].}\NormalTok{head}\OperatorTok{].}\NormalTok{w}\OperatorTok{+}\NormalTok{a}\OperatorTok{[}\NormalTok{i}\OperatorTok{].}\NormalTok{w}\OperatorTok{\});}
        \OperatorTok{\}}
\NormalTok{        ans}\OperatorTok{=}\NormalTok{max}\OperatorTok{(}\NormalTok{ans}\OperatorTok{,}\NormalTok{s}\OperatorTok{[}\NormalTok{i}\OperatorTok{].}\NormalTok{v}\OperatorTok{[}\NormalTok{s}\OperatorTok{[}\NormalTok{i}\OperatorTok{].}\NormalTok{tail}\OperatorTok{].}\NormalTok{w}\OperatorTok{);}
    \OperatorTok{\}}
    \ControlFlowTok{for}\OperatorTok{(}\DataTypeTok{int}\NormalTok{ i}\OperatorTok{=}\DecValTok{1}\OperatorTok{;}\NormalTok{i}\OperatorTok{\textless{}=}\NormalTok{n}\OperatorTok{;}\NormalTok{i}\OperatorTok{++)}\NormalTok{s}\OperatorTok{[}\NormalTok{i}\OperatorTok{].}\NormalTok{clear}\OperatorTok{();}
    \ControlFlowTok{for}\OperatorTok{(}\DataTypeTok{int}\NormalTok{ i}\OperatorTok{=}\NormalTok{n}\OperatorTok{;}\NormalTok{i}\OperatorTok{\textgreater{}=}\DecValTok{1}\OperatorTok{;}\NormalTok{i}\OperatorTok{{-}{-})\{}
\NormalTok{        s}\OperatorTok{[}\NormalTok{i}\OperatorTok{].}\NormalTok{push}\OperatorTok{((}\NormalTok{node}\OperatorTok{)\{{-}}\DecValTok{1}\OperatorTok{,}\NormalTok{a}\OperatorTok{[}\NormalTok{i}\OperatorTok{].}\NormalTok{w}\OperatorTok{\});}
        \ControlFlowTok{for}\OperatorTok{(}\DataTypeTok{int}\NormalTok{ j}\OperatorTok{=}\NormalTok{i}\OperatorTok{+}\DecValTok{1}\OperatorTok{;}\NormalTok{j}\OperatorTok{\textless{}=}\NormalTok{n}\OperatorTok{;}\NormalTok{j}\OperatorTok{++)\{}
\NormalTok{            s}\OperatorTok{[}\NormalTok{j}\OperatorTok{].}\NormalTok{pop}\OperatorTok{(}\NormalTok{a}\OperatorTok{[}\NormalTok{j}\OperatorTok{].}\NormalTok{x}\OperatorTok{{-}}\NormalTok{a}\OperatorTok{[}\NormalTok{i}\OperatorTok{].}\NormalTok{x}\OperatorTok{);}
\NormalTok{            s}\OperatorTok{[}\NormalTok{i}\OperatorTok{].}\NormalTok{push}\OperatorTok{((}\NormalTok{node}\OperatorTok{)\{}\NormalTok{a}\OperatorTok{[}\NormalTok{j}\OperatorTok{].}\NormalTok{x}\OperatorTok{{-}}\NormalTok{a}\OperatorTok{[}\NormalTok{i}\OperatorTok{].}\NormalTok{x}\OperatorTok{,}\NormalTok{s}\OperatorTok{[}\NormalTok{j}\OperatorTok{].}\NormalTok{v}\OperatorTok{[}\NormalTok{s}\OperatorTok{[}\NormalTok{j}\OperatorTok{].}\NormalTok{head}\OperatorTok{].}\NormalTok{w}\OperatorTok{+}\NormalTok{a}\OperatorTok{[}\NormalTok{i}\OperatorTok{].}\NormalTok{w}\OperatorTok{\});}
        \OperatorTok{\}}
\NormalTok{        ans}\OperatorTok{=}\NormalTok{max}\OperatorTok{(}\NormalTok{ans}\OperatorTok{,}\NormalTok{s}\OperatorTok{[}\NormalTok{i}\OperatorTok{].}\NormalTok{v}\OperatorTok{[}\NormalTok{s}\OperatorTok{[}\NormalTok{i}\OperatorTok{].}\NormalTok{tail}\OperatorTok{].}\NormalTok{w}\OperatorTok{);}
    \OperatorTok{\}}
\NormalTok{    cout}\OperatorTok{\textless{}\textless{}}\NormalTok{ans}\OperatorTok{\textless{}\textless{}}\CharTok{\textquotesingle{}}\SpecialCharTok{\textbackslash{}n}\CharTok{\textquotesingle{}}\OperatorTok{;}
    \ControlFlowTok{return} \DecValTok{0}\OperatorTok{;}
\OperatorTok{\}}
\end{Highlighting}
\end{Shaded}

\subsubsection{3.2. 斜率优化DP}

\paragraph{3.2.1. 斜率优化}

可以应用斜率优化的动态规划模型往往某一维度的状态数为 \(O(n)\)
级别,而为找到最优转移点,单次的状态转移需考察 \(O(n)\)
个子阶段,这使该维度的转移的时间复杂度开销达到 \(O(n^2)\)。

斜率优化可以在将转移式变形后,通过考察几何意义,将可能成为最优转移点的集合缩小到一个凸包上,再根据题目条件将转移的时间复杂度缩小到
\(O(n)\) 或 \(O(n\log n)\)。

\paragraph{应用实例1 {[}HNOI2008{]} 玩具装箱}

\subparagraph{题目描述}

有 \(n\) 个玩具,第 \(i\) 个玩具价值为 \(c_i\)。要求将这 \(n\)
个玩具排成一排,分成若干段。对于一段 \([l,r]\),它的代价为: \[
(r-l+\sum_{i=l}^r c_i-L)^2
\] 求分段的最小代价。

\(1\le n\le 5\times 10^4,1\le L,0\le c_i\le 10^7\)

\subparagraph{解题思路}

令 \(f_i\) 表示前 \(i\) 个物品,分若干段的最小代价。有状态转移方程: \[
f_i=\min_{j<i}{\{f_j+(pre_i-pre_j+i-j-1-L)^2\}}
\] 其中 \(pre_i = \sum_{j=1}^i c_j\)。直接转移时间复杂度为
\(O (n^2)\),无法解决本题。

为简化状态转移方程式,令 \(s_i=pre_i+i,L’=L+1\),则 \[
f_i=\min_{j<i}\{f_j+(s_i-s_j-L’)^2\}
\] 设 \(j\) 为使 \(f_i\) 最小的转移点,有: \[
f_i=f_j+(s_i-s_j-L’)^2
\] 考虑一次函数的斜截式 \(y=kx+b\),将方程转化为这个形式。

其中变量 \(x, y\) 与 \(j\) 有关,\(b, k\) 与 \(i\) 有关,且要求 \(x\) 随
\(j\) 单调递增,仅 \(b\) 中包含 \(f_i\)。

按照上面的规则,对方程进行整理得到: \[
f_j+(s_j+L’)^2=2s_i(s_j+L)+f_i-{s_i}^2
\]

\[
\left\{
\begin{array}{lr}
y = f_j+(s_j+L’)^2\\
k = 2s_i\\
x = s_j+L\\
b = f_i-{s_i}^2
\end{array}
\right.
\]

\((x_j, y_j)\) 的几何意义为直线 \(y=kx+b\)
上的一个点,又因为转移时目的是最小化 \(f_i\),在上面的表示当中,\(f_i\)
只与直线的截距 \(b\) 有关。所以问题可转化为如何选取 \(j\) ,使得过点
\((x_j, y_j)\) 的直线的截距 \(b\) 最小。

注意到直线的斜率不变,相当于平移直线 \(y=kx\) ,直到其经过图中的一个点。

\includegraphics{https://cdn.tonyyin.top/2021/02/25/39c3696467e33.png}

于是可以在转移的同时维护 \((x_i,y_i)\)
构成的凸包,利用单调性二分得到时间复杂度为 \(O(n\log n)\) 的算法。发现
\(k=2s_i\) 关于 \(i\)
单调递增,因此可以利用单调队列来维护得到时间复杂度为 \(O(n)\) 的算法。

\subparagraph{代码实现}

\begin{Shaded}
\begin{Highlighting}[]
\PreprocessorTok{\#include }\ImportTok{\textless{}bits/stdc++.h\textgreater{}}
\PreprocessorTok{\#define int }\DataTypeTok{long}\PreprocessorTok{ }\DataTypeTok{long}
\KeywordTok{using} \KeywordTok{namespace}\NormalTok{ std}\OperatorTok{;}
\AttributeTok{const} \DataTypeTok{int}\NormalTok{ MAXN }\OperatorTok{=} \FloatTok{5e4} \OperatorTok{+} \DecValTok{10}\OperatorTok{;}
\DataTypeTok{int}\NormalTok{ n}\OperatorTok{,}\NormalTok{ L}\OperatorTok{;}
\DataTypeTok{int}\NormalTok{ c}\OperatorTok{[}\NormalTok{MAXN}\OperatorTok{],}\NormalTok{ s}\OperatorTok{[}\NormalTok{MAXN}\OperatorTok{],}\NormalTok{ f}\OperatorTok{[}\NormalTok{MAXN}\OperatorTok{];}
\DataTypeTok{double}\NormalTok{ slope}\OperatorTok{(}\DataTypeTok{int}\NormalTok{ i}\OperatorTok{,} \DataTypeTok{int}\NormalTok{ j}\OperatorTok{)} \OperatorTok{\{}
    \ControlFlowTok{return} \OperatorTok{((}\NormalTok{f}\OperatorTok{[}\NormalTok{j}\OperatorTok{]+(}\NormalTok{s}\OperatorTok{[}\NormalTok{j}\OperatorTok{]+}\NormalTok{L}\OperatorTok{)*(}\NormalTok{s}\OperatorTok{[}\NormalTok{j}\OperatorTok{]+}\NormalTok{L}\OperatorTok{))} \OperatorTok{{-}} \OperatorTok{(}\NormalTok{f}\OperatorTok{[}\NormalTok{i}\OperatorTok{]+(}\NormalTok{s}\OperatorTok{[}\NormalTok{i}\OperatorTok{]+}\NormalTok{L}\OperatorTok{)*(}\NormalTok{s}\OperatorTok{[}\NormalTok{i}\OperatorTok{]+}\NormalTok{L}\OperatorTok{)))} \OperatorTok{/} \OperatorTok{(}\DataTypeTok{double}\OperatorTok{)(}\NormalTok{s}\OperatorTok{[}\NormalTok{j}\OperatorTok{]} \OperatorTok{{-}}\NormalTok{ s}\OperatorTok{[}\NormalTok{i}\OperatorTok{]);}
\OperatorTok{\}}
\DataTypeTok{int}\NormalTok{ head}\OperatorTok{,}\NormalTok{ tail}\OperatorTok{,}\NormalTok{ q}\OperatorTok{[}\NormalTok{MAXN}\OperatorTok{];}
\DataTypeTok{signed}\NormalTok{ main}\OperatorTok{()} \OperatorTok{\{}
\NormalTok{    scanf}\OperatorTok{(}\StringTok{"}\SpecialCharTok{\%lld\%lld}\StringTok{"}\OperatorTok{,} \OperatorTok{\&}\NormalTok{n}\OperatorTok{,} \OperatorTok{\&}\NormalTok{L}\OperatorTok{);}
\NormalTok{    L }\OperatorTok{+=} \DecValTok{1}\OperatorTok{;}
    \ControlFlowTok{for}\OperatorTok{(}\DataTypeTok{int}\NormalTok{ i }\OperatorTok{=} \DecValTok{1}\OperatorTok{;}\NormalTok{ i }\OperatorTok{\textless{}=}\NormalTok{ n}\OperatorTok{;}\NormalTok{ i}\OperatorTok{++)} \OperatorTok{\{}
\NormalTok{        scanf}\OperatorTok{(}\StringTok{"}\SpecialCharTok{\%lld}\StringTok{"}\OperatorTok{,} \OperatorTok{\&}\NormalTok{c}\OperatorTok{[}\NormalTok{i}\OperatorTok{]);}
\NormalTok{        s}\OperatorTok{[}\NormalTok{i}\OperatorTok{]} \OperatorTok{=}\NormalTok{ s}\OperatorTok{[}\NormalTok{i }\OperatorTok{{-}} \DecValTok{1}\OperatorTok{]} \OperatorTok{+}\NormalTok{ c}\OperatorTok{[}\NormalTok{i}\OperatorTok{]} \OperatorTok{+} \DecValTok{1}\OperatorTok{;}
    \OperatorTok{\}}
\NormalTok{    head }\OperatorTok{=}\NormalTok{ tail }\OperatorTok{=} \DecValTok{1}\OperatorTok{;}
    \ControlFlowTok{for}\OperatorTok{(}\DataTypeTok{int}\NormalTok{ i }\OperatorTok{=} \DecValTok{1}\OperatorTok{;}\NormalTok{ i }\OperatorTok{\textless{}=}\NormalTok{ n}\OperatorTok{;}\NormalTok{ i}\OperatorTok{++)} \OperatorTok{\{}
        \ControlFlowTok{while}\OperatorTok{(}\NormalTok{head }\OperatorTok{\textless{}}\NormalTok{ tail }\OperatorTok{\&\&}\NormalTok{ slope}\OperatorTok{(}\NormalTok{q}\OperatorTok{[}\NormalTok{head}\OperatorTok{],}\NormalTok{ q}\OperatorTok{[}\NormalTok{head }\OperatorTok{+} \DecValTok{1}\OperatorTok{])} \OperatorTok{\textless{}=} \DecValTok{2} \OperatorTok{*}\NormalTok{ s}\OperatorTok{[}\NormalTok{i}\OperatorTok{])} \OperatorTok{\{}
\NormalTok{            head}\OperatorTok{++;}
        \OperatorTok{\}}
\NormalTok{        f}\OperatorTok{[}\NormalTok{i}\OperatorTok{]} \OperatorTok{=}\NormalTok{ f}\OperatorTok{[}\NormalTok{q}\OperatorTok{[}\NormalTok{head}\OperatorTok{]]} \OperatorTok{+} \OperatorTok{(}\NormalTok{s}\OperatorTok{[}\NormalTok{i}\OperatorTok{]} \OperatorTok{{-}}\NormalTok{ s}\OperatorTok{[}\NormalTok{q}\OperatorTok{[}\NormalTok{head}\OperatorTok{]]} \OperatorTok{{-}}\NormalTok{ L}\OperatorTok{)} \OperatorTok{*} \OperatorTok{(}\NormalTok{s}\OperatorTok{[}\NormalTok{i}\OperatorTok{]} \OperatorTok{{-}}\NormalTok{ s}\OperatorTok{[}\NormalTok{q}\OperatorTok{[}\NormalTok{head}\OperatorTok{]]} \OperatorTok{{-}}\NormalTok{ L}\OperatorTok{);}
        \ControlFlowTok{while}\OperatorTok{(}\NormalTok{head }\OperatorTok{\textless{}}\NormalTok{ tail }\OperatorTok{\&\&}\NormalTok{ slope}\OperatorTok{(}\NormalTok{q}\OperatorTok{[}\NormalTok{tail }\OperatorTok{{-}} \DecValTok{1}\OperatorTok{],}\NormalTok{ q}\OperatorTok{[}\NormalTok{tail}\OperatorTok{])} \OperatorTok{\textgreater{}=}\NormalTok{ slope}\OperatorTok{(}\NormalTok{q}\OperatorTok{[}\NormalTok{tail}\OperatorTok{],}\NormalTok{ i}\OperatorTok{))} \OperatorTok{\{}
\NormalTok{            tail}\OperatorTok{{-}{-};}
        \OperatorTok{\}}
\NormalTok{        q}\OperatorTok{[++}\NormalTok{tail}\OperatorTok{]} \OperatorTok{=}\NormalTok{ i}\OperatorTok{;}
    \OperatorTok{\}}
\NormalTok{    printf}\OperatorTok{(}\StringTok{"}\SpecialCharTok{\%lld}\StringTok{"}\OperatorTok{,}\NormalTok{ f}\OperatorTok{[}\NormalTok{n}\OperatorTok{]);}
    \ControlFlowTok{return} \DecValTok{0}\OperatorTok{;}
\OperatorTok{\}}
\end{Highlighting}
\end{Shaded}

\subsubsection{3.3. 四边形不等式优化 DP}

\paragraph{3.3.1. 四边形不等式}

若对于任意整数 \(l_1,l_2,r_2,r_1\) 满足
\(l_1\le l_2\le r_2\le r_1\),若二元函数 \(w(x,y)\) 满足 \[
w(l_1,r_1)+w(l_2,r_2)\ge w(l_1,r_2)+w(l_2,r_1)\tag{1}
\] 则称 \(w\) 满足四边形不等式。四边形不等式有一个经典的等价形式: \[
w(l,r+1)+w(l+1,r)\ge w(l,r)+w(l+1,r+1)\tag{2}
\] 其中
\(l\lt r\)。由(1)得到(2)较为显然,由(2)推出(1)可以通过数学归纳法证得。

假设对于 \(l+x\lt r\) 有 \[
w(l,r+1)+w(l+x,r)\ge w(l,r)+w(l+x,r+1)\\
\] 在 \(x=1\) 时此式为(1)。

对于 \(l+x+1\lt r\),根据(2)有 \[
w(l+x,r+1)+w(l+x+1,r)\ge w(l+x,r)+w(l+x+1,r+1)\\
\] 两式相加可得 \[
w(l,r+1)+w(l+x+1,r)\ge w(l,r)+w(l+x+1,r+1)
\] 同理可得 \(w(l,r+y)+w(l+x,r)\ge w(l,r)+w(l+x,r+y)\),即证。

\paragraph{3.3.2. 四边形不等式对 DP 的优化}

四边形不等式优化 DP 最早在 1971 年由 Donald E. Knuth. 提出,而后又在
1980 年由 F. Frances Yao 进行了深入研究和系统化整理。

\subparagraph{3.3.2.1. 对一维 DP 的优化}

在优化一维 DP 时,DP 的形式往往为
\(f[i]=\min_{0\le j\lt i}\{f[j]+w(j,i)\}\)。不妨设令 \(f[i]\)
取得最小值的 \(j\) 为 \(d[i]\),那么若 \(w\) 满足四边形不等式,则 \(d\)
单调不减,称 \(f\) 具有决策单调性。利用此性质可以使时间复杂度为
\(O(n^2)\) 的计算简化为 \(O(n\log n)\)。本文将不对这类优化进行细致探讨。

\subparagraph{3.3.2.2. 对二维 DP 的优化}

在优化二维 DP 时,DP 的形式则多为
\(f[i][j]=\min_{i\le k\lt j}\{f[i][k]+f[k+1][j]+w(i,j)\}(i\lt j)\),\(f[i][i]=0\)。

不妨设令 \(f[i][j]\) 取得最小值的 \(k\) 为 \(d[i][j]\),那么

\begin{itemize}
\tightlist
\item
  若

  \begin{enumerate}
  \def\labelenumi{\arabic{enumi}.}
  \tightlist
  \item
    \(w\) 满足四边形不等式,
  \item
    对于任意 \(l_1\le l_2\le r_2\le r_1\),有
    \(w(l_1,r_1)\ge w(l_2,r_2)\);
  \end{enumerate}
\item
  则

  \begin{enumerate}
  \def\labelenumi{\arabic{enumi}.}
  \tightlist
  \item
    \(dp\) 也满足四边形不等式,
  \item
    \(d[i][j]\le d[i+1][j]\)、\(d[i][j]\le d[i][j+1]\)。
  \end{enumerate}
\end{itemize}

对于满足 \(d[i][j]\le \min\{d[i+1][j],d[i][j+1]\}\) 的,我们称 \(dp\)
具有二维决策单调性。在对 \(f[i][j]\) 进行计算时,仅考察位于
\([d[i][j-1],d[i+1][j]]\) 中的 \(k\) 可以使时间复杂度为 \(O(n^3)\)
的计算优化为 \(O(n^2)\)。

(1) \(dp\) 满足四边形不等式的证明

关于 \(dp\) 满足四边形不等式的证明,考虑使用数学归纳法。

当 \(r-l=1\) 时, \[
dp[l][r+1]+dp[l+1][r]=dp[l][l+2]\ge w(l,l+2)\\
dp[l][r]+dp[l+1][r+1]=dp[l][l+1]+dp[l+1][l+2]=w(l,l+1)+w(l+1,l+2)\\
dp[l][r+1]+dp[l+1][r]\ge dp[l][r]+dp[l+1][r+1]
\] 设在 \(r-l\lt k\) 时,\(dp\) 满足四边形不等式。为方便起见,设
\(x=d[l][r+1],y=d[l+1][r]\),那么 \[
dp[l][r+1]+dp[l+1][r]=(dp[l][x]+dp[x+1][r+1]+w(l,r+1))+(dp[l+1][y]+dp[y+1][r]+w(l+1,r))\\
dp[l][r]+dp[l+1][r+1]\le(dp[l][x]+dp[x+1][r]+w(l,r))+(dp[l+1][y]+dp[y+1][r+1]+w(l+1,r+1))\\
\] 欲证 \(dp[l][r+1]+dp[l+1][r]\ge dp[l][r]+dp[l+1][r+1]\),只需证 \[
dp[x+1][r+1]+dp[y+1][r]+w(l,r+1)+w(l+1,r)\ge dp[x+1][r]+dp[y+1][r+1]+w(l,r)+w(l+1,r+1)
\] 只需证 \[
dp[x+1][r+1]+dp[y+1][r]\ge dp[x+1][r]+dp[y+1][r+1]\\
w(l,r+1)+w(l+1,r)\ge w(l,r)+w(l+1,r+1)
\] 两式分别可以由归纳假设以及 \(w\) 满足四边形不等式得到,即证 \(dp\)
满足四边形不等式。

(2) 二维决策单调性的证明

关于二维决策单调性的证明,设 \(x=d[i][j]\),\(i\le k\lt x\)。根据 \(dp\)
满足四边形不等式和 \(d[i][j]\) 的最优性,有 \[
dp[i][x]+dp[i+1][k]\ge dp[i][k]+dp[i+1][x]\\
dp[i][k]+dp[k+1][j]\ge dp[i][x]+dp[x+1][j]
\] 可以得到 \[
dp[i+1][k]+dp[k+1][j]\ge dp[i+1][x]+dp[x+1][j]
\] 因此对于 \(dp[i+1][j]\) 的转移,\(k\in [i,d[i][j])\) 一定不优于
\(d[i][j]\),有 \(d[i+1][j]\ge d[i][j]\)。

类似的,设 \(x\lt k\le j\),有 \[
dp[x+1][j]+dp[k+1][j-1]\ge dp[x+1][j-1]+dp[k+1][j]\\
dp[i][k]+dp[k+1][j]\ge dp[i][x]+dp[x+1][j]
\] 可以得到 \[
dp[i][k]+dp[k+1][j-1]\ge dp[i][x]+dp[x+1][j-1]
\] 即 \(d[i][j-1]\le d[i][j]\)。

(3) 时间复杂度的证明

关于时间复杂度是 \(O(n^2)\) 的证明,考虑 \[
\sum_{i=1}^{n-1}\sum_{j=i+1}^n(d[i+1][j]-d[i][j-1])\\
=\sum_{i=2}^n\sum_{j=i+1}^nd[i][j]-\sum_{i=1}^{n-1}\sum_{j=i}^{n-1}d[i][j]\\
=\sum_{i=1}^{n-1}d[i][n]-\sum_{j=2}^nd[1][j]-\sum_{i=1}^nd[i][i]
\] 并且 \(d[i][j]\in[1,n]\),因此
\(O(\sum_{i=1}^{n-1}\sum_{j=i+1}^n(d[i+1][j]-d[i][j-1]))=O(n^2)\)。

\subparagraph{应用实例1 石子合并}

题目描述

有 \(n\) 堆直线排列的石子,第 \(i\) 堆石子有 \(a[i]\)
个。规定每次只能合并任意相邻的两堆石子,并产生两堆石子数量之和的疲劳值。现在要将石子有序的合并成一堆,试求最小总疲劳值。

\(1\le n\le 5000\)

解题思路

设 \(dp[i][j]\) 为将下标在 \([i,j]\)
内的石子合并为一堆所需的最小疲劳值,\(w(i,j)=\sum_{k=i}^ja[k]\),那么有
\[
dp[i][j]=
\begin{cases}
\min_{i\le k\lt j}\{dp[i][k]+dp[k+1][j]+w(i,j)\}&i\lt j\\[2ex]
0&i=j
\end{cases}
\] 直接转移时间复杂度为 \(O(n^3)\),不能接受。发现 \(w\) 满足 \[
w(l,r+1)+w(l+1,r)=w(l,r)+w(l+1,r+1)
\] 即复合四边形不等式条件,因此可以对上述 DP 进行优化,达到 \(O(n^2)\)
的时间复杂度。

代码实现

\begin{Shaded}
\begin{Highlighting}[]
\PreprocessorTok{\#include }\ImportTok{\textless{}bits/stdc++.h\textgreater{}}
\KeywordTok{using} \KeywordTok{namespace}\NormalTok{ std}\OperatorTok{;}
\AttributeTok{const} \DataTypeTok{int}\NormalTok{ maxn}\OperatorTok{=}\DecValTok{5005}\OperatorTok{;}
\DataTypeTok{int}\NormalTok{ n}\OperatorTok{,}\NormalTok{a}\OperatorTok{[}\NormalTok{maxn}\OperatorTok{],}\NormalTok{s}\OperatorTok{[}\NormalTok{maxn}\OperatorTok{],}\NormalTok{dp}\OperatorTok{[}\NormalTok{maxn}\OperatorTok{][}\NormalTok{maxn}\OperatorTok{],}\NormalTok{d}\OperatorTok{[}\NormalTok{maxn}\OperatorTok{][}\NormalTok{maxn}\OperatorTok{];}
\PreprocessorTok{\#define w}\OperatorTok{(}\NormalTok{i}\OperatorTok{,}\NormalTok{j}\OperatorTok{)}\PreprocessorTok{ }\OperatorTok{(}\NormalTok{s}\OperatorTok{[}\NormalTok{j}\OperatorTok{]{-}}\NormalTok{s}\OperatorTok{[}\NormalTok{i}\OperatorTok{{-}}\DecValTok{1}\OperatorTok{])}
\DataTypeTok{int}\NormalTok{ main}\OperatorTok{()\{}
\NormalTok{    cin}\OperatorTok{\textgreater{}\textgreater{}}\NormalTok{n}\OperatorTok{;}
    \ControlFlowTok{for}\OperatorTok{(}\DataTypeTok{int}\NormalTok{ i}\OperatorTok{=}\DecValTok{1}\OperatorTok{;}\NormalTok{i}\OperatorTok{\textless{}=}\NormalTok{n}\OperatorTok{;}\NormalTok{i}\OperatorTok{++)}\NormalTok{cin}\OperatorTok{\textgreater{}\textgreater{}}\NormalTok{a}\OperatorTok{[}\NormalTok{i}\OperatorTok{];}
    \ControlFlowTok{for}\OperatorTok{(}\DataTypeTok{int}\NormalTok{ i}\OperatorTok{=}\DecValTok{1}\OperatorTok{;}\NormalTok{i}\OperatorTok{\textless{}=}\NormalTok{n}\OperatorTok{;}\NormalTok{i}\OperatorTok{++)}\NormalTok{s}\OperatorTok{[}\NormalTok{i}\OperatorTok{]=}\NormalTok{s}\OperatorTok{[}\NormalTok{i}\OperatorTok{{-}}\DecValTok{1}\OperatorTok{]+}\NormalTok{a}\OperatorTok{[}\NormalTok{i}\OperatorTok{];}
\NormalTok{    memset}\OperatorTok{(}\NormalTok{dp}\OperatorTok{,}\BaseNTok{0x3f}\OperatorTok{,}\KeywordTok{sizeof}\OperatorTok{(}\NormalTok{dp}\OperatorTok{));}
    \ControlFlowTok{for}\OperatorTok{(}\DataTypeTok{int}\NormalTok{ i}\OperatorTok{=}\DecValTok{1}\OperatorTok{;}\NormalTok{i}\OperatorTok{\textless{}=}\NormalTok{n}\OperatorTok{;}\NormalTok{i}\OperatorTok{++)}\NormalTok{dp}\OperatorTok{[}\NormalTok{i}\OperatorTok{][}\NormalTok{i}\OperatorTok{]=}\DecValTok{0}\OperatorTok{,}\NormalTok{d}\OperatorTok{[}\NormalTok{i}\OperatorTok{][}\NormalTok{i}\OperatorTok{]=}\NormalTok{i}\OperatorTok{;}
    \ControlFlowTok{for}\OperatorTok{(}\DataTypeTok{int}\NormalTok{ l}\OperatorTok{=}\DecValTok{2}\OperatorTok{;}\NormalTok{l}\OperatorTok{\textless{}=}\NormalTok{n}\OperatorTok{;}\NormalTok{l}\OperatorTok{++)\{}
        \ControlFlowTok{for}\OperatorTok{(}\DataTypeTok{int}\NormalTok{ i}\OperatorTok{=}\DecValTok{1}\OperatorTok{,}\NormalTok{j}\OperatorTok{;}\NormalTok{i}\OperatorTok{+}\NormalTok{l}\OperatorTok{{-}}\DecValTok{1}\OperatorTok{\textless{}=}\NormalTok{n}\OperatorTok{;}\NormalTok{i}\OperatorTok{++)\{}
\NormalTok{            j}\OperatorTok{=}\NormalTok{i}\OperatorTok{+}\NormalTok{l}\OperatorTok{{-}}\DecValTok{1}\OperatorTok{;}
            \ControlFlowTok{for}\OperatorTok{(}\DataTypeTok{int}\NormalTok{ k}\OperatorTok{=}\NormalTok{d}\OperatorTok{[}\NormalTok{i}\OperatorTok{][}\NormalTok{j}\OperatorTok{{-}}\DecValTok{1}\OperatorTok{];}\NormalTok{k}\OperatorTok{\textless{}=}\NormalTok{min}\OperatorTok{(}\NormalTok{j}\OperatorTok{{-}}\DecValTok{1}\OperatorTok{,}\NormalTok{d}\OperatorTok{[}\NormalTok{i}\OperatorTok{+}\DecValTok{1}\OperatorTok{][}\NormalTok{j}\OperatorTok{]);}\NormalTok{k}\OperatorTok{++)\{}
                \DataTypeTok{int}\NormalTok{ nx}\OperatorTok{=}\NormalTok{dp}\OperatorTok{[}\NormalTok{i}\OperatorTok{][}\NormalTok{k}\OperatorTok{]+}\NormalTok{dp}\OperatorTok{[}\NormalTok{k}\OperatorTok{+}\DecValTok{1}\OperatorTok{][}\NormalTok{j}\OperatorTok{]+}\NormalTok{w}\OperatorTok{(}\NormalTok{i}\OperatorTok{,}\NormalTok{j}\OperatorTok{);}
                \ControlFlowTok{if}\OperatorTok{(}\NormalTok{nx}\OperatorTok{\textless{}}\NormalTok{dp}\OperatorTok{[}\NormalTok{i}\OperatorTok{][}\NormalTok{j}\OperatorTok{])\{}
\NormalTok{                    dp}\OperatorTok{[}\NormalTok{i}\OperatorTok{][}\NormalTok{j}\OperatorTok{]=}\NormalTok{nx}\OperatorTok{;}
\NormalTok{                    d}\OperatorTok{[}\NormalTok{i}\OperatorTok{][}\NormalTok{j}\OperatorTok{]=}\NormalTok{k}\OperatorTok{;}
                \OperatorTok{\}}
            \OperatorTok{\}}
        \OperatorTok{\}}
    \OperatorTok{\}}
\NormalTok{    cout}\OperatorTok{\textless{}\textless{}}\NormalTok{dp}\OperatorTok{[}\DecValTok{1}\OperatorTok{][}\NormalTok{n}\OperatorTok{]\textless{}\textless{}}\NormalTok{endl}\OperatorTok{;}
    \ControlFlowTok{return} \DecValTok{0}\OperatorTok{;}
\OperatorTok{\}}
\end{Highlighting}
\end{Shaded}

\subparagraph{应用实例2 {[}IOI2000{]} 邮局}

题目描述

给定 \(n\) 个村庄在一条直线上的坐标。现在要选一些村庄建立 \(k\)
个邮局,使得每个村庄与其最近的邮局之间的距离总和最小。试求这个最小距离和。

\(1\le k\le 300\),\(k\le n\le 3000\),\(1\le\) 村庄位置 \(\le 10000\)。

解题思路

首先将村庄按其在坐标轴上的位置排序,设第 \(i\) 个村庄的坐标为 \(x_i\)。

设 \(dp[i][j]\) 表示前 \(i\) 个村庄建立 \(j\)
个邮局的最小距离和,\(w(l,r)\) 表示在第 \(p\) 个村庄建立一个邮局所得到的
\(\sum_{i=l}^r|x_i-x_p|\) 的最小值。不难看出
\(p=\lfloor \frac{l+r}{2}\rfloor\),因此 \[
w(l,r)=(\sum_{i=p+1}^rx_i-x_p(r-p))+(x_p(p-l)-\sum_{i=l}^{p-1}x_i)\\
w(l,r)=w(l,r-1)+x_r-x_p\\
w(l,r)=w(l+1,r)+x_p-x_l\\
\] 可以在 \(dp\) 转移前 \(O(n^2)\) 递推,也可以使用前缀和每次转移时
\(O(1)\) 计算。

同时 DP 递推式有 \[
dp[i][j]=
\begin{cases}
\min_{1\le k\le i}\{dp[k-1][j-1]+w(k,i)\}&i\not=0,j\not=0\\[2ex]
1&i=0,j=0\\[2ex]
0&else
\end{cases}
\] 直接递推是 \(O(n^2k)\) 的,不能接受。发现 \[
\begin{aligned}
w(l,r)+w(l+1,r-1)&=2w(l+1,r-1)+x_r-x_l\\
w(l+1,r)+w(l,r-1)&=2w(l+1,r-1)+x_r-x_l-(x_{\lfloor \frac{l+r+1}{2}\rfloor}-x_{\lfloor \frac{l+r-1}{2}\rfloor})
\end{aligned}
\] 因为 \(x\) 递增,有
\((x_{\lfloor \frac{l+r+1}{2}\rfloor}-x_{\lfloor \frac{l+r-1}{2}\rfloor})\ge0\),所以
\(w(l,r)+w(l+1,r-1)\ge w(l+1,r)+w(l,r-1)\),\(w\)
满足四边形不等式,\(dp\)
满足四边形不等式。因此对转移时计算的区间进行优化就能达到 \(O(nk)\)
的时间复杂度。

代码实现

\begin{Shaded}
\begin{Highlighting}[]
\PreprocessorTok{\#include }\ImportTok{\textless{}bits/stdc++.h\textgreater{}}
\KeywordTok{using} \KeywordTok{namespace}\NormalTok{ std}\OperatorTok{;}
\KeywordTok{typedef} \DataTypeTok{long} \DataTypeTok{long}\NormalTok{ ll}\OperatorTok{;}
\AttributeTok{const} \DataTypeTok{int}\NormalTok{ maxn}\OperatorTok{=}\FloatTok{3e3}\OperatorTok{+}\DecValTok{5}\OperatorTok{,}\NormalTok{maxp}\OperatorTok{=}\DecValTok{305}\OperatorTok{;}
\DataTypeTok{int}\NormalTok{ n}\OperatorTok{,}\NormalTok{p}\OperatorTok{,}\NormalTok{dp}\OperatorTok{[}\NormalTok{maxn}\OperatorTok{][}\NormalTok{maxp}\OperatorTok{],}\NormalTok{d}\OperatorTok{[}\NormalTok{maxn}\OperatorTok{][}\NormalTok{maxp}\OperatorTok{],}\NormalTok{x}\OperatorTok{[}\NormalTok{maxn}\OperatorTok{],}\NormalTok{s}\OperatorTok{[}\NormalTok{maxn}\OperatorTok{];}
\KeywordTok{inline} \DataTypeTok{int}\NormalTok{ w}\OperatorTok{(}\DataTypeTok{int}\NormalTok{ l}\OperatorTok{,}\DataTypeTok{int}\NormalTok{ r}\OperatorTok{)\{}
    \DataTypeTok{int}\NormalTok{ mid}\OperatorTok{=}\NormalTok{l}\OperatorTok{+}\NormalTok{r}\OperatorTok{\textgreater{}\textgreater{}}\DecValTok{1}\OperatorTok{;}
    \DataTypeTok{int}\NormalTok{ res}\OperatorTok{=}\NormalTok{s}\OperatorTok{[}\NormalTok{r}\OperatorTok{]{-}}\NormalTok{s}\OperatorTok{[}\NormalTok{mid}\OperatorTok{]{-}}\NormalTok{x}\OperatorTok{[}\NormalTok{mid}\OperatorTok{]*(}\NormalTok{r}\OperatorTok{{-}}\NormalTok{mid}\OperatorTok{);}
\NormalTok{    res}\OperatorTok{+=}\NormalTok{x}\OperatorTok{[}\NormalTok{mid}\OperatorTok{]*(}\NormalTok{mid}\OperatorTok{{-}}\NormalTok{l}\OperatorTok{){-}(}\NormalTok{s}\OperatorTok{[}\NormalTok{mid}\OperatorTok{{-}}\DecValTok{1}\OperatorTok{]{-}}\NormalTok{s}\OperatorTok{[}\NormalTok{l}\OperatorTok{{-}}\DecValTok{1}\OperatorTok{]);}
    \ControlFlowTok{return}\NormalTok{ res}\OperatorTok{;}
\OperatorTok{\}}
\DataTypeTok{int}\NormalTok{ main}\OperatorTok{()\{}
\NormalTok{    cin}\OperatorTok{\textgreater{}\textgreater{}}\NormalTok{n}\OperatorTok{\textgreater{}\textgreater{}}\NormalTok{p}\OperatorTok{;}
    \ControlFlowTok{for}\OperatorTok{(}\DataTypeTok{int}\NormalTok{ i}\OperatorTok{=}\DecValTok{1}\OperatorTok{;}\NormalTok{i}\OperatorTok{\textless{}=}\NormalTok{n}\OperatorTok{;}\NormalTok{i}\OperatorTok{++)}\NormalTok{cin}\OperatorTok{\textgreater{}\textgreater{}}\NormalTok{x}\OperatorTok{[}\NormalTok{i}\OperatorTok{];}
\NormalTok{    sort}\OperatorTok{(}\NormalTok{x}\OperatorTok{+}\DecValTok{1}\OperatorTok{,}\NormalTok{x}\OperatorTok{+}\DecValTok{1}\OperatorTok{+}\NormalTok{n}\OperatorTok{);}
    \ControlFlowTok{for}\OperatorTok{(}\DataTypeTok{int}\NormalTok{ i}\OperatorTok{=}\DecValTok{1}\OperatorTok{;}\NormalTok{i}\OperatorTok{\textless{}=}\NormalTok{n}\OperatorTok{;}\NormalTok{i}\OperatorTok{++)}\NormalTok{s}\OperatorTok{[}\NormalTok{i}\OperatorTok{]=}\NormalTok{s}\OperatorTok{[}\NormalTok{i}\OperatorTok{{-}}\DecValTok{1}\OperatorTok{]+}\NormalTok{x}\OperatorTok{[}\NormalTok{i}\OperatorTok{];}
    \ControlFlowTok{for}\OperatorTok{(}\DataTypeTok{int}\NormalTok{ i}\OperatorTok{=}\DecValTok{1}\OperatorTok{;}\NormalTok{i}\OperatorTok{\textless{}=}\NormalTok{p}\OperatorTok{;}\NormalTok{i}\OperatorTok{++)}\NormalTok{d}\OperatorTok{[}\NormalTok{n}\OperatorTok{+}\DecValTok{1}\OperatorTok{][}\NormalTok{i}\OperatorTok{]=}\NormalTok{n}\OperatorTok{;}
\NormalTok{    memset}\OperatorTok{(}\NormalTok{dp}\OperatorTok{,}\BaseNTok{0x3f}\OperatorTok{,}\KeywordTok{sizeof}\OperatorTok{(}\NormalTok{dp}\OperatorTok{));}
\NormalTok{    dp}\OperatorTok{[}\DecValTok{0}\OperatorTok{][}\DecValTok{0}\OperatorTok{]=}\DecValTok{0}\OperatorTok{;}
    \ControlFlowTok{for}\OperatorTok{(}\DataTypeTok{int}\NormalTok{ j}\OperatorTok{=}\DecValTok{1}\OperatorTok{;}\NormalTok{j}\OperatorTok{\textless{}=}\NormalTok{p}\OperatorTok{;}\NormalTok{j}\OperatorTok{++)}
        \ControlFlowTok{for}\OperatorTok{(}\DataTypeTok{int}\NormalTok{ i}\OperatorTok{=}\NormalTok{n}\OperatorTok{;}\NormalTok{i}\OperatorTok{\textgreater{}=}\DecValTok{1}\OperatorTok{;}\NormalTok{i}\OperatorTok{{-}{-})}
        \ControlFlowTok{for}\OperatorTok{(}\DataTypeTok{int}\NormalTok{ k}\OperatorTok{=}\NormalTok{d}\OperatorTok{[}\NormalTok{i}\OperatorTok{][}\NormalTok{j}\OperatorTok{{-}}\DecValTok{1}\OperatorTok{];}\NormalTok{k}\OperatorTok{\textless{}=}\NormalTok{d}\OperatorTok{[}\NormalTok{i}\OperatorTok{+}\DecValTok{1}\OperatorTok{][}\NormalTok{j}\OperatorTok{];}\NormalTok{k}\OperatorTok{++)\{}
            \DataTypeTok{int}\NormalTok{ nx}\OperatorTok{=}\NormalTok{dp}\OperatorTok{[}\NormalTok{k}\OperatorTok{][}\NormalTok{j}\OperatorTok{{-}}\DecValTok{1}\OperatorTok{]+}\NormalTok{w}\OperatorTok{(}\NormalTok{k}\OperatorTok{+}\DecValTok{1}\OperatorTok{,}\NormalTok{i}\OperatorTok{);}
            \ControlFlowTok{if}\OperatorTok{(}\NormalTok{nx}\OperatorTok{\textless{}}\NormalTok{dp}\OperatorTok{[}\NormalTok{i}\OperatorTok{][}\NormalTok{j}\OperatorTok{])}
\NormalTok{                dp}\OperatorTok{[}\NormalTok{i}\OperatorTok{][}\NormalTok{j}\OperatorTok{]=}\NormalTok{nx}\OperatorTok{,}\NormalTok{d}\OperatorTok{[}\NormalTok{i}\OperatorTok{][}\NormalTok{j}\OperatorTok{]=}\NormalTok{k}\OperatorTok{;}
        \OperatorTok{\}}
\NormalTok{    cout}\OperatorTok{\textless{}\textless{}}\NormalTok{dp}\OperatorTok{[}\NormalTok{n}\OperatorTok{][}\NormalTok{p}\OperatorTok{]\textless{}\textless{}}\CharTok{\textquotesingle{}}\SpecialCharTok{\textbackslash{}n}\CharTok{\textquotesingle{}}\OperatorTok{;}
    \ControlFlowTok{return} \DecValTok{0}\OperatorTok{;}
\OperatorTok{\}}
\end{Highlighting}
\end{Shaded}

\subsubsection{3.4. CDQ 分治优化 DP}

\paragraph{3.4.1. CDQ 分治}

CDQ
分治最早为陈丹琦在《从\textless cash\textgreater 谈一类分治算法的应用》(2008国家集训队作业)中整理的,因此以
CDQ 命名。陈丹琦本人整理的 CDQ
分治应用方法主要为先将在线维护的问题转换为离线问题,并添加一个操作时间的维度,再对高维的偏序问题进行分治处理。后来引申的
CDQ
分治应用范围则包括了可离线的高维偏序问题以及转移包括高维偏序条件的动态规划问题。下文将探讨
CDQ 分治对转移包括高维偏序条件的动态规划问题的优化。

\paragraph{应用实例1 长者}

\subparagraph{题目描述}

给定一个长度为 \(n\) 的排列
\(a\),其中有若干位置上的数字已经确定了,剩下位置上的数字不确定。

你需要钦定未被确定的位置上的数字,使得得到的排列的最长上升子序列(LIS,Longest
Increasing Subsequence)长度尽量长。试求出这个最长长度。

\(n\leq 10^5\)

\subparagraph{解题思路}

注意到 ``是一个排列''
这个限制,事实上仅仅限制了那些未被确定的位置上的数字不能与已经确定的位置相同,而我们在计算最优答案时并不需要特别处理未被确定的数字之间是否相同,因为
LIS 中不能出现两个相同的数字,因此剩下位置上出现相同数字一定不优。

设 \(f(i,j)\) 表示对于 \(a\) 数组长度为 \(i\)
的前缀,在所有钦定方案中,以 \(j\) 结尾的最长的 LIS 长度。有转移:

\begin{itemize}
\item
  若位置 \(i\) 被确定了,则
  \(f(i,x)=\begin{cases}\max_{j\lt p_i}(f(i-1,j))+1&x=p_i\\0&x\not=p_i\end{cases}\);
\item
  否则 \(f(i,j)=\max\{\max_{k\lt j}(f(i-1,k))+1,f(i-1,j)\}\),这里的
  \(j\) 必须不在已经确定的数字中出现过。
\end{itemize}

这样直接转移是 \(O(n^2)\) 的。

但是二维 dp 是不方便使用 CDQ 分治优化的,仍然考虑转成一维 dp 的形式。

设 \(f(i)\) 表示以位置 \(i\) 结尾的最长 LIS,此处位置 \(i\)
的数字必须被确定了。

则枚举 LIS 中上一个有确定数字的位置,可以得到:

\[
f(i)=\max_{p_j\lt p_i}(f(j)+\min\{cnt[i]-cnt[j],rem[p_i]-rem[p_j]\})+1
\] 其中,\(cnt[i]\) 表示 \(a\) 长度为 \(i\)
的前缀中没被确定的位置个数,\(rem[p_i]\) 表示 \(< p_i\)
且未出现在被确定的位置的数字个数。

注意到这里同时有 \(p_j<p_i\) 和 \(j<i\) 两个限制,且中间的 \(\min\)
可以分类讨论为两种情况:

\begin{enumerate}
\def\labelenumi{\arabic{enumi}.}
\tightlist
\item
  \(cnt[i]-cnt[j]>=rem[p_i]-rem[p_j] \iff cnt[i]-rem[p_i]>=cnt[j]-rem[p_j]\)
\item
  \(cnt[i]-cnt[j]<rem[p_i]-rem[p_j] \iff cnt[i]-rem[p_i]<cnt[j]-rem[p_j]\)
\end{enumerate}

这也就相当于第三个限制,加上前两个就是三维偏序。因此我们先按下标分治,然后左右两边分别按
\(cnt[i]-rem[p_i]\) 排序,以第一种情况为例,双指针扫的时候用树状数组维护
\(f(j)-rem[p_j]\) 即可。

整体时间复杂度 \(O(n \log^2 n)\),空间复杂度 \(O(n)\)。

\subparagraph{代码实现}

\begin{Shaded}
\begin{Highlighting}[]
\CommentTok{//a[i].x=id,a[i].y=cnt[i]{-}rem[p[i]],a[i].tp=(p[i]\textgreater{}0);}
\DataTypeTok{void}\NormalTok{ cdq}\OperatorTok{(}\DataTypeTok{int}\NormalTok{ l}\OperatorTok{,}\DataTypeTok{int}\NormalTok{ r}\OperatorTok{)\{}
    \ControlFlowTok{if}\OperatorTok{(}\NormalTok{l}\OperatorTok{\textgreater{}=}\NormalTok{r}\OperatorTok{)}\ControlFlowTok{return} \OperatorTok{;}
    \DataTypeTok{int}\NormalTok{ mid}\OperatorTok{=(}\NormalTok{l}\OperatorTok{+}\NormalTok{r}\OperatorTok{)\textgreater{}\textgreater{}}\DecValTok{1}\OperatorTok{;}
\NormalTok{    cdq}\OperatorTok{(}\NormalTok{l}\OperatorTok{,}\NormalTok{mid}\OperatorTok{);}
\NormalTok{    sort}\OperatorTok{(}\NormalTok{a}\OperatorTok{+}\NormalTok{l}\OperatorTok{,}\NormalTok{a}\OperatorTok{+}\NormalTok{mid}\OperatorTok{+}\DecValTok{1}\OperatorTok{,}\NormalTok{cmpy}\OperatorTok{);}
\NormalTok{    sort}\OperatorTok{(}\NormalTok{a}\OperatorTok{+}\NormalTok{mid}\OperatorTok{+}\DecValTok{1}\OperatorTok{,}\NormalTok{a}\OperatorTok{+}\NormalTok{r}\OperatorTok{+}\DecValTok{1}\OperatorTok{,}\NormalTok{cmpy}\OperatorTok{);}
    \DataTypeTok{int}\NormalTok{ i}\OperatorTok{=}\NormalTok{l}\OperatorTok{,}\NormalTok{j}\OperatorTok{=}\NormalTok{mid}\OperatorTok{+}\DecValTok{1}\OperatorTok{;}
    \ControlFlowTok{for}\OperatorTok{(;}\NormalTok{j}\OperatorTok{\textless{}=}\NormalTok{r}\OperatorTok{;}\NormalTok{j}\OperatorTok{++)\{}
        \ControlFlowTok{if}\OperatorTok{(!}\NormalTok{a}\OperatorTok{[}\NormalTok{j}\OperatorTok{].}\NormalTok{tp}\OperatorTok{)}\ControlFlowTok{continue}\OperatorTok{;}
        \ControlFlowTok{while}\OperatorTok{(}\NormalTok{i}\OperatorTok{\textless{}=}\NormalTok{mid}\OperatorTok{\&\&}\NormalTok{a}\OperatorTok{[}\NormalTok{i}\OperatorTok{].}\NormalTok{y}\OperatorTok{\textless{}=}\NormalTok{a}\OperatorTok{[}\NormalTok{j}\OperatorTok{].}\NormalTok{y}\OperatorTok{)}
            \OperatorTok{\{}\ControlFlowTok{if}\OperatorTok{(}\NormalTok{a}\OperatorTok{[}\NormalTok{i}\OperatorTok{].}\NormalTok{tp}\OperatorTok{)}\NormalTok{upd}\OperatorTok{(}\NormalTok{a}\OperatorTok{[}\NormalTok{i}\OperatorTok{].}\NormalTok{p}\OperatorTok{,}\NormalTok{a}\OperatorTok{[}\NormalTok{i}\OperatorTok{].}\NormalTok{f}\OperatorTok{{-}}\NormalTok{rem}\OperatorTok{[}\NormalTok{a}\OperatorTok{[}\NormalTok{i}\OperatorTok{].}\NormalTok{p}\OperatorTok{]);}\NormalTok{i}\OperatorTok{++;\}}
\NormalTok{        a}\OperatorTok{[}\NormalTok{j}\OperatorTok{].}\NormalTok{f}\OperatorTok{=}\NormalTok{max}\OperatorTok{(}\NormalTok{a}\OperatorTok{[}\NormalTok{j}\OperatorTok{].}\NormalTok{f}\OperatorTok{,}\NormalTok{qry}\OperatorTok{(}\NormalTok{a}\OperatorTok{[}\NormalTok{j}\OperatorTok{].}\NormalTok{p}\OperatorTok{)+}\NormalTok{rem}\OperatorTok{[}\NormalTok{a}\OperatorTok{[}\NormalTok{j}\OperatorTok{].}\NormalTok{p}\OperatorTok{]+}\DecValTok{1}\OperatorTok{);}
    \OperatorTok{\}}\ControlFlowTok{for}\OperatorTok{(}\NormalTok{j}\OperatorTok{=}\NormalTok{l}\OperatorTok{;}\NormalTok{j}\OperatorTok{\textless{}}\NormalTok{i}\OperatorTok{;}\NormalTok{j}\OperatorTok{++)}\ControlFlowTok{if}\OperatorTok{(}\NormalTok{a}\OperatorTok{[}\NormalTok{j}\OperatorTok{].}\NormalTok{tp}\OperatorTok{)}\NormalTok{clr}\OperatorTok{(}\NormalTok{a}\OperatorTok{[}\NormalTok{j}\OperatorTok{].}\NormalTok{p}\OperatorTok{);}
\NormalTok{    i}\OperatorTok{=}\NormalTok{mid}\OperatorTok{;}\NormalTok{j}\OperatorTok{=}\NormalTok{r}\OperatorTok{;}
    \ControlFlowTok{for}\OperatorTok{(;}\NormalTok{j}\OperatorTok{\textgreater{}}\NormalTok{mid}\OperatorTok{;}\NormalTok{j}\OperatorTok{{-}{-})\{}
        \ControlFlowTok{while}\OperatorTok{(}\NormalTok{i}\OperatorTok{\textgreater{}=}\NormalTok{l}\OperatorTok{\&\&}\NormalTok{a}\OperatorTok{[}\NormalTok{i}\OperatorTok{].}\NormalTok{y}\OperatorTok{\textgreater{}}\NormalTok{a}\OperatorTok{[}\NormalTok{j}\OperatorTok{].}\NormalTok{y}\OperatorTok{)}
            \OperatorTok{\{}\ControlFlowTok{if}\OperatorTok{(}\NormalTok{a}\OperatorTok{[}\NormalTok{i}\OperatorTok{].}\NormalTok{tp}\OperatorTok{)}\NormalTok{upd}\OperatorTok{(}\NormalTok{a}\OperatorTok{[}\NormalTok{i}\OperatorTok{].}\NormalTok{p}\OperatorTok{,}\NormalTok{a}\OperatorTok{[}\NormalTok{i}\OperatorTok{].}\NormalTok{f}\OperatorTok{{-}}\NormalTok{cnt}\OperatorTok{[}\NormalTok{a}\OperatorTok{[}\NormalTok{i}\OperatorTok{].}\NormalTok{p}\OperatorTok{]);}\NormalTok{i}\OperatorTok{++;\}}
\NormalTok{        a}\OperatorTok{[}\NormalTok{j}\OperatorTok{].}\NormalTok{f}\OperatorTok{=}\NormalTok{max}\OperatorTok{(}\NormalTok{a}\OperatorTok{[}\NormalTok{j}\OperatorTok{].}\NormalTok{f}\OperatorTok{,}\NormalTok{qry}\OperatorTok{(}\NormalTok{a}\OperatorTok{[}\NormalTok{j}\OperatorTok{].}\NormalTok{p}\OperatorTok{)+}\NormalTok{cnt}\OperatorTok{[}\NormalTok{a}\OperatorTok{[}\NormalTok{j}\OperatorTok{].}\NormalTok{p}\OperatorTok{]+}\DecValTok{1}\OperatorTok{);}
    \OperatorTok{\}}\ControlFlowTok{for}\OperatorTok{(}\NormalTok{j}\OperatorTok{=}\NormalTok{mid}\OperatorTok{;}\NormalTok{j}\OperatorTok{\textgreater{}}\NormalTok{i}\OperatorTok{;}\NormalTok{j}\OperatorTok{{-}{-})}\ControlFlowTok{if}\OperatorTok{(}\NormalTok{a}\OperatorTok{[}\NormalTok{j}\OperatorTok{].}\NormalTok{tp}\OperatorTok{)}\NormalTok{clr}\OperatorTok{(}\NormalTok{a}\OperatorTok{[}\NormalTok{j}\OperatorTok{].}\NormalTok{p}\OperatorTok{);}
\NormalTok{    sort}\OperatorTok{(}\NormalTok{a}\OperatorTok{+}\NormalTok{l}\OperatorTok{,}\NormalTok{a}\OperatorTok{+}\NormalTok{r}\OperatorTok{+}\DecValTok{1}\OperatorTok{,}\NormalTok{cmpx}\OperatorTok{);}
\NormalTok{    cdq}\OperatorTok{(}\NormalTok{mid}\OperatorTok{+}\DecValTok{1}\OperatorTok{,}\NormalTok{r}\OperatorTok{);}
\OperatorTok{\}}
\end{Highlighting}
\end{Shaded}

\paragraph{应用实例2 {[}CEOI2017{]} Building Bridges}

\subparagraph{题目描述}

给定 \(n\) 个柱子,每个柱子用 \((h_i,w_i)\) 描述。拆除第 \(i\)
根柱子的代价为 \(w_i\)。在 \(i,j\) 之间架桥的代价为
\((h_i-h_j)^2\),同时还需要拆除 \([i+1,j-1]\)
之间的所有柱子。求通过桥梁把 \(1\) 和 \(n\) 连通的最小代价。

\(2\leq n\leq 10^5\),\(0\leq h_i,|w_i|\leq 10^6\)

\subparagraph{解题思路}

不妨设 \(dp[i]\) 为通过桥梁将 \(1\) 和 \(i\)
联通的最小代价,\(s_i=\sum_{i\in[1,i]}w_i\),不难得到 \[
dp[i]=\min_{j\lt i}(dp[j]+s_{i-1}-s_j+(h_i-h_j)^2)
\] 将 \(\min\) 去掉可以得到 \[
(dp[j]-s_j+h_j^2)=h_i\times(2h_j)-(s_{i-1}+h_i^2)+dp[i]
\]

发现能斜率优化,然而直接进行斜率优化需要满足斜率 \(2h_j\) 单调,因此考虑
CDQ 分治。具体来讲,以第一维为下标,计算 \(cdq(l,r)\) 时,令
\(mid=\lfloor\frac{l+r}{2}\rfloor\),对 \([l,mid]\) 中的下标 \(i\)
直接计算出由 \((dp[i]-s_i+h_i^2,h_i)\) 构成的下凸包 ,\([mid+1,r]\)
中的元素按照 \(h\) 排序然后做斜率优化 DP 计算贡献即可。

\subparagraph{代码实现}

\begin{Shaded}
\begin{Highlighting}[]
\PreprocessorTok{\#include }\ImportTok{\textless{}bits/stdc++.h\textgreater{}}
\KeywordTok{using} \KeywordTok{namespace}\NormalTok{ std}\OperatorTok{;}
\KeywordTok{typedef} \DataTypeTok{long} \DataTypeTok{long}\NormalTok{ ll}\OperatorTok{;}
\AttributeTok{const} \DataTypeTok{int}\NormalTok{ maxn}\OperatorTok{=}\FloatTok{1e5}\OperatorTok{+}\DecValTok{5}\OperatorTok{;}
\AttributeTok{const} \DataTypeTok{double}\NormalTok{ eps}\OperatorTok{=}\FloatTok{1e{-}6}\OperatorTok{;}
\KeywordTok{inline} \DataTypeTok{int}\NormalTok{ read}\OperatorTok{()\{}
    \DataTypeTok{int}\NormalTok{ x}\OperatorTok{=}\DecValTok{0}\OperatorTok{,}\NormalTok{f}\OperatorTok{=}\DecValTok{1}\OperatorTok{,}\NormalTok{ch}\OperatorTok{=}\NormalTok{getchar}\OperatorTok{();}
    \ControlFlowTok{while}\OperatorTok{(}\NormalTok{ch}\OperatorTok{\textless{}}\CharTok{\textquotesingle{}0\textquotesingle{}}\OperatorTok{||}\NormalTok{ch}\OperatorTok{\textgreater{}}\CharTok{\textquotesingle{}9\textquotesingle{}}\OperatorTok{)\{}\ControlFlowTok{if}\OperatorTok{(}\NormalTok{ch}\OperatorTok{==}\CharTok{\textquotesingle{}{-}\textquotesingle{}}\OperatorTok{)}\NormalTok{f}\OperatorTok{={-}}\DecValTok{1}\OperatorTok{;}\NormalTok{ch}\OperatorTok{=}\NormalTok{getchar}\OperatorTok{();\}}
    \ControlFlowTok{while}\OperatorTok{(}\NormalTok{ch}\OperatorTok{\textgreater{}=}\CharTok{\textquotesingle{}0\textquotesingle{}}\OperatorTok{\&\&}\NormalTok{ch}\OperatorTok{\textless{}=}\CharTok{\textquotesingle{}9\textquotesingle{}}\OperatorTok{)\{}\NormalTok{x}\OperatorTok{=(}\NormalTok{x}\OperatorTok{\textless{}\textless{}}\DecValTok{3}\OperatorTok{)+(}\NormalTok{x}\OperatorTok{\textless{}\textless{}}\DecValTok{1}\OperatorTok{)+}\NormalTok{ch}\OperatorTok{{-}}\DecValTok{48}\OperatorTok{;}\NormalTok{ch}\OperatorTok{=}\NormalTok{getchar}\OperatorTok{();\}}
    \ControlFlowTok{return}\NormalTok{ x}\OperatorTok{*}\NormalTok{f}\OperatorTok{;}
\OperatorTok{\}}
\DataTypeTok{int}\NormalTok{ n}\OperatorTok{;}
\NormalTok{ll f}\OperatorTok{[}\NormalTok{maxn}\OperatorTok{],}\NormalTok{s}\OperatorTok{[}\NormalTok{maxn}\OperatorTok{];}
\KeywordTok{struct}\NormalTok{ node}\OperatorTok{\{}\DataTypeTok{int}\NormalTok{ h}\OperatorTok{,}\NormalTok{id}\OperatorTok{;}\NormalTok{ll s}\OperatorTok{;\}}\NormalTok{a}\OperatorTok{[}\NormalTok{maxn}\OperatorTok{];}
\DataTypeTok{bool}\NormalTok{ cmph}\OperatorTok{(}\NormalTok{node a}\OperatorTok{,}\NormalTok{node b}\OperatorTok{)\{}\ControlFlowTok{return}\NormalTok{ a}\OperatorTok{.}\NormalTok{h}\OperatorTok{\textless{}}\NormalTok{b}\OperatorTok{.}\NormalTok{h}\OperatorTok{;\}}
\DataTypeTok{bool}\NormalTok{ cmpid}\OperatorTok{(}\NormalTok{node a}\OperatorTok{,}\NormalTok{node b}\OperatorTok{)\{}\ControlFlowTok{return}\NormalTok{ a}\OperatorTok{.}\NormalTok{id}\OperatorTok{\textless{}}\NormalTok{b}\OperatorTok{.}\NormalTok{id}\OperatorTok{;\}}
\DataTypeTok{int}\NormalTok{ q}\OperatorTok{[}\NormalTok{maxn}\OperatorTok{],}\NormalTok{hd}\OperatorTok{=}\DecValTok{1}\OperatorTok{,}\NormalTok{tl}\OperatorTok{;}
\DataTypeTok{double}\NormalTok{ slope}\OperatorTok{(}\DataTypeTok{int}\NormalTok{ u}\OperatorTok{,}\DataTypeTok{int}\NormalTok{ v}\OperatorTok{)\{}
\NormalTok{    ll uy}\OperatorTok{=}\NormalTok{f}\OperatorTok{[}\NormalTok{a}\OperatorTok{[}\NormalTok{u}\OperatorTok{].}\NormalTok{id}\OperatorTok{]{-}}\NormalTok{a}\OperatorTok{[}\NormalTok{u}\OperatorTok{].}\NormalTok{s}\OperatorTok{+(}\DecValTok{1}\BuiltInTok{ll}\OperatorTok{)*}\NormalTok{a}\OperatorTok{[}\NormalTok{u}\OperatorTok{].}\NormalTok{h}\OperatorTok{*}\NormalTok{a}\OperatorTok{[}\NormalTok{u}\OperatorTok{].}\NormalTok{h}\OperatorTok{;}
\NormalTok{    ll vy}\OperatorTok{=}\NormalTok{f}\OperatorTok{[}\NormalTok{a}\OperatorTok{[}\NormalTok{v}\OperatorTok{].}\NormalTok{id}\OperatorTok{]{-}}\NormalTok{a}\OperatorTok{[}\NormalTok{v}\OperatorTok{].}\NormalTok{s}\OperatorTok{+(}\DecValTok{1}\BuiltInTok{ll}\OperatorTok{)*}\NormalTok{a}\OperatorTok{[}\NormalTok{v}\OperatorTok{].}\NormalTok{h}\OperatorTok{*}\NormalTok{a}\OperatorTok{[}\NormalTok{v}\OperatorTok{].}\NormalTok{h}\OperatorTok{;}
    \ControlFlowTok{if}\OperatorTok{(}\NormalTok{a}\OperatorTok{[}\NormalTok{u}\OperatorTok{].}\NormalTok{h}\OperatorTok{==}\NormalTok{a}\OperatorTok{[}\NormalTok{v}\OperatorTok{].}\NormalTok{h}\OperatorTok{)\{}\ControlFlowTok{return}\NormalTok{ vy}\OperatorTok{\textgreater{}}\NormalTok{uy}\OperatorTok{?}\FloatTok{1e18}\OperatorTok{:{-}}\FloatTok{1e18}\OperatorTok{;\}}
    \ControlFlowTok{return} \OperatorTok{(}\NormalTok{uy}\OperatorTok{{-}}\NormalTok{vy}\OperatorTok{)*}\FloatTok{1.0}\OperatorTok{/(}\NormalTok{a}\OperatorTok{[}\NormalTok{u}\OperatorTok{].}\NormalTok{h}\OperatorTok{{-}}\NormalTok{a}\OperatorTok{[}\NormalTok{v}\OperatorTok{].}\NormalTok{h}\OperatorTok{);}
\OperatorTok{\}}
\DataTypeTok{void}\NormalTok{ cdq}\OperatorTok{(}\DataTypeTok{int}\NormalTok{ l}\OperatorTok{,}\DataTypeTok{int}\NormalTok{ r}\OperatorTok{)\{}
    \ControlFlowTok{if}\OperatorTok{(}\NormalTok{l}\OperatorTok{\textgreater{}=}\NormalTok{r}\OperatorTok{)}\ControlFlowTok{return} \OperatorTok{;}
    \DataTypeTok{int}\NormalTok{ mid}\OperatorTok{=}\NormalTok{l}\OperatorTok{+}\NormalTok{r}\OperatorTok{\textgreater{}\textgreater{}}\DecValTok{1}\OperatorTok{;}
\NormalTok{    cdq}\OperatorTok{(}\NormalTok{l}\OperatorTok{,}\NormalTok{mid}\OperatorTok{);}
\NormalTok{    sort}\OperatorTok{(}\NormalTok{a}\OperatorTok{+}\NormalTok{l}\OperatorTok{,}\NormalTok{a}\OperatorTok{+}\NormalTok{mid}\OperatorTok{+}\DecValTok{1}\OperatorTok{,}\NormalTok{cmph}\OperatorTok{);}
\NormalTok{    sort}\OperatorTok{(}\NormalTok{a}\OperatorTok{+}\NormalTok{mid}\OperatorTok{+}\DecValTok{1}\OperatorTok{,}\NormalTok{a}\OperatorTok{+}\NormalTok{r}\OperatorTok{+}\DecValTok{1}\OperatorTok{,}\NormalTok{cmph}\OperatorTok{);}
    \CommentTok{//斜率优化}
    \ControlFlowTok{for}\OperatorTok{(}\DataTypeTok{int}\NormalTok{ i}\OperatorTok{=}\NormalTok{l}\OperatorTok{;}\NormalTok{i}\OperatorTok{\textless{}=}\NormalTok{mid}\OperatorTok{;}\NormalTok{i}\OperatorTok{++)\{}
        \ControlFlowTok{while}\OperatorTok{(}\NormalTok{tl}\OperatorTok{\textgreater{}}\DecValTok{1}\OperatorTok{\&\&}\NormalTok{slope}\OperatorTok{(}\NormalTok{q}\OperatorTok{[}\NormalTok{tl}\OperatorTok{],}\NormalTok{i}\OperatorTok{)\textless{}}\NormalTok{slope}\OperatorTok{(}\NormalTok{q}\OperatorTok{[}\NormalTok{tl}\OperatorTok{{-}}\DecValTok{1}\OperatorTok{],}\NormalTok{q}\OperatorTok{[}\NormalTok{tl}\OperatorTok{]))}\NormalTok{tl}\OperatorTok{{-}{-};}
\NormalTok{        q}\OperatorTok{[++}\NormalTok{tl}\OperatorTok{]=}\NormalTok{i}\OperatorTok{;}
    \OperatorTok{\}}
    \ControlFlowTok{for}\OperatorTok{(}\DataTypeTok{int}\NormalTok{ j}\OperatorTok{=}\NormalTok{mid}\OperatorTok{+}\DecValTok{1}\OperatorTok{;}\NormalTok{j}\OperatorTok{\textless{}=}\NormalTok{r}\OperatorTok{;}\NormalTok{j}\OperatorTok{++)\{}
        \ControlFlowTok{while}\OperatorTok{(}\NormalTok{hd}\OperatorTok{\textless{}}\NormalTok{tl}\OperatorTok{\&\&}\NormalTok{slope}\OperatorTok{(}\NormalTok{q}\OperatorTok{[}\NormalTok{hd}\OperatorTok{],}\NormalTok{q}\OperatorTok{[}\NormalTok{hd}\OperatorTok{+}\DecValTok{1}\OperatorTok{])\textless{}=}\FloatTok{2.0}\OperatorTok{*}\NormalTok{a}\OperatorTok{[}\NormalTok{j}\OperatorTok{].}\NormalTok{h}\OperatorTok{+}\NormalTok{eps}\OperatorTok{)}\NormalTok{hd}\OperatorTok{++;}
        \ControlFlowTok{if}\OperatorTok{(}\NormalTok{hd}\OperatorTok{\textless{}=}\NormalTok{tl}\OperatorTok{)}\NormalTok{f}\OperatorTok{[}\NormalTok{a}\OperatorTok{[}\NormalTok{j}\OperatorTok{].}\NormalTok{id}\OperatorTok{]=}\NormalTok{min}\OperatorTok{(}\NormalTok{f}\OperatorTok{[}\NormalTok{a}\OperatorTok{[}\NormalTok{j}\OperatorTok{].}\NormalTok{id}\OperatorTok{],}\NormalTok{s}\OperatorTok{[}\NormalTok{a}\OperatorTok{[}\NormalTok{j}\OperatorTok{].}\NormalTok{id}\OperatorTok{{-}}\DecValTok{1}\OperatorTok{]+(}\DecValTok{1}\BuiltInTok{ll}\OperatorTok{)*}\NormalTok{a}\OperatorTok{[}\NormalTok{j}\OperatorTok{].}\NormalTok{h}\OperatorTok{*}\NormalTok{a}\OperatorTok{[}\NormalTok{j}\OperatorTok{].}\NormalTok{h}\OperatorTok{+}\NormalTok{f}\OperatorTok{[}\NormalTok{a}\OperatorTok{[}\NormalTok{q}\OperatorTok{[}\NormalTok{hd}\OperatorTok{]].}\NormalTok{id}\OperatorTok{]{-}}\NormalTok{a}\OperatorTok{[}\NormalTok{q}\OperatorTok{[}\NormalTok{hd}\OperatorTok{]].}\NormalTok{s}\OperatorTok{+(}\DecValTok{1}\BuiltInTok{ll}\OperatorTok{)*}\NormalTok{a}\OperatorTok{[}\NormalTok{q}\OperatorTok{[}\NormalTok{hd}\OperatorTok{]].}\NormalTok{h}\OperatorTok{*}\NormalTok{a}\OperatorTok{[}\NormalTok{q}\OperatorTok{[}\NormalTok{hd}\OperatorTok{]].}\NormalTok{h}\OperatorTok{{-}}\DecValTok{2}\BuiltInTok{ll}\OperatorTok{*}\NormalTok{a}\OperatorTok{[}\NormalTok{j}\OperatorTok{].}\NormalTok{h}\OperatorTok{*}\NormalTok{a}\OperatorTok{[}\NormalTok{q}\OperatorTok{[}\NormalTok{hd}\OperatorTok{]].}\NormalTok{h}\OperatorTok{);}
    \OperatorTok{\}}\NormalTok{hd}\OperatorTok{=}\DecValTok{1}\OperatorTok{,}\NormalTok{tl}\OperatorTok{=}\DecValTok{0}\OperatorTok{;}
\NormalTok{    sort}\OperatorTok{(}\NormalTok{a}\OperatorTok{+}\NormalTok{mid}\OperatorTok{+}\DecValTok{1}\OperatorTok{,}\NormalTok{a}\OperatorTok{+}\NormalTok{r}\OperatorTok{+}\DecValTok{1}\OperatorTok{,}\NormalTok{cmpid}\OperatorTok{);}
\NormalTok{    cdq}\OperatorTok{(}\NormalTok{mid}\OperatorTok{+}\DecValTok{1}\OperatorTok{,}\NormalTok{r}\OperatorTok{);}
\OperatorTok{\}}
\DataTypeTok{int}\NormalTok{ main}\OperatorTok{()\{}
\NormalTok{    n}\OperatorTok{=}\NormalTok{read}\OperatorTok{();}
    \ControlFlowTok{for}\OperatorTok{(}\DataTypeTok{int}\NormalTok{ i}\OperatorTok{=}\DecValTok{1}\OperatorTok{;}\NormalTok{i}\OperatorTok{\textless{}=}\NormalTok{n}\OperatorTok{;}\NormalTok{i}\OperatorTok{++)}\NormalTok{a}\OperatorTok{[}\NormalTok{i}\OperatorTok{].}\NormalTok{h}\OperatorTok{=}\NormalTok{read}\OperatorTok{(),}\NormalTok{a}\OperatorTok{[}\NormalTok{i}\OperatorTok{].}\NormalTok{id}\OperatorTok{=}\NormalTok{i}\OperatorTok{;}
    \ControlFlowTok{for}\OperatorTok{(}\DataTypeTok{int}\NormalTok{ i}\OperatorTok{=}\DecValTok{1}\OperatorTok{;}\NormalTok{i}\OperatorTok{\textless{}=}\NormalTok{n}\OperatorTok{;}\NormalTok{i}\OperatorTok{++)}\NormalTok{a}\OperatorTok{[}\NormalTok{i}\OperatorTok{].}\NormalTok{s}\OperatorTok{=}\NormalTok{a}\OperatorTok{[}\NormalTok{i}\OperatorTok{{-}}\DecValTok{1}\OperatorTok{].}\NormalTok{s}\OperatorTok{+}\NormalTok{read}\OperatorTok{(),}\NormalTok{s}\OperatorTok{[}\NormalTok{i}\OperatorTok{]=}\NormalTok{a}\OperatorTok{[}\NormalTok{i}\OperatorTok{].}\NormalTok{s}\OperatorTok{;}
\NormalTok{    memset}\OperatorTok{(}\NormalTok{f}\OperatorTok{,}\BaseNTok{0x3f}\OperatorTok{,}\KeywordTok{sizeof}\OperatorTok{(}\NormalTok{f}\OperatorTok{));}\NormalTok{f}\OperatorTok{[}\DecValTok{1}\OperatorTok{]=}\DecValTok{0}\OperatorTok{;}
\NormalTok{    cdq}\OperatorTok{(}\DecValTok{1}\OperatorTok{,}\NormalTok{n}\OperatorTok{);}
\NormalTok{    cout}\OperatorTok{\textless{}\textless{}}\NormalTok{f}\OperatorTok{[}\NormalTok{n}\OperatorTok{]\textless{}\textless{}}\CharTok{\textquotesingle{}}\SpecialCharTok{\textbackslash{}n}\CharTok{\textquotesingle{}}\OperatorTok{;}
    \ControlFlowTok{return} \DecValTok{0}\OperatorTok{;}
\OperatorTok{\}}
\end{Highlighting}
\end{Shaded}

\subsection{4. 结语}

在解决动态规划问题时,人们常常会受算法难度和思路的限制。经过此次学习,本组详细整理并掌握了两种复杂的动态规划算法以及四种可以被广泛应用的算法优化方式,形成文字化资料,可以被后人利用。而其中本组所附上的典例解析可以作为参考资料,有利于后人对于该算法的学习,可以更好地理解并应用该算法。
通过本次课题研究,我们对文献法的掌握更加深入。本组在后续的成文过程中也参考了各资料的成文方式,使我们的文章深入浅出,更为简洁易懂。本小组成员各司其职,分工合理。这使得我们的研学速度快,成果质量高。我们的心态也逐渐变得稳健,铸就了迎难而上,不放弃的精神。
本次研学我们的创新意识稍有不足,主要进行了资料的整合而非独立,全新的创作。以后需加以改进。

\subsection{5. 参考资料}

\subsection{6. 附件}
